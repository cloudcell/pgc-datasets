\input style
\chapter{4 семантическая характеристика языков 
программирования}
в ПРЕДЫДУЩЕЙ ГЛАВЕ МЫ ВЫДВИНУЛИ ТЕЗИС, ЧТО ЗНАЕМ СЕМАНТИКУ 
КОНСТРУКЦИИ $S$ ДОСТАТОЧНО ХОРОШО, ЕСЛИ ЗНАЕМ ЕЕ 
"ПРЕОБРАЗОВАТЕЛЬ ПРЕДИКАТОВ", Т. Е. ПРАВИЛО, УКАЗЫВАЮЩЕЕ НАМ, 
КАК ВЫВЕСТИ ПО ЛЮБОМУ ПОСТУСЛОВИЮ $R$ СООТВЕТСТВУЮЩЕЕ 
СЛАБЕЙШЕЕ ПРЕДУСЛОВИЕ (КОТОРОЕ МЫ ОБОЗНАЧИЛИ ЧЕРЕЗ 
$\wp(S, R))$, ХАРАКТЕРИЗУЮЩЕЕ ТЕ НАЧАЛЬНЫЕ СОСТОЯНИЯ, ПРИ 
КОТОРЫХ ЗАПУСК ПРИВЕДЕТ К СОБЫТИЮ ПРАВИЛЬНОГО ЗАВЕРШЕНИЯ, 
ПРИЧЕМ СИСТЕМА ОСТАНЕТСЯ В КОНЕЧНОМ СОСТОЯНИИ, 
УДОВЛЕТВОРЯЮЩЕМ ПОСТУСЛОВИЮ $R$. вОПРОС В ТОМ, КАК ВЫВОДИТЬ 
$\wp(S, R)$ ДЛЯ ЗАДАННЫХ $S$ И $R$.

оСТАВИМ ПОКА ВОПРОС ОБ ОДИНОЧНОЙ КОНКРЕТНОЙ КОНСТРУКЦИИ $S$. 
пРОГРАММА, НАПИСАННАЯ НА ХОРОШО ОПРЕДЕЛЕННОМ ЯЗЫКЕ 
ПРОГРАММИРОВАНИЯ, МОЖЕТ РАССМАТРИВАТЬСЯ КАК НЕКАЯ 
КОНСТРУКЦИЯ, ТАКАЯ КОНСТРУКЦИЯ, КОТОРУЮ МЫ ЗНАЕМ ДОСТАТОЧНО 
ХОРОШО, ЕСЛИ ЗНАЕМ СООТВЕТСТВУЮЩИЙ ПРЕОБРАЗОВАТЕЛЬ ПРЕДИКАТОВ 
ОДНАКО ЯЗЫК ПРОГРАММИРОВАНИЯ ПОЛЕЗЕН ТОЛЬКО ПРИ ТОМ УСЛОВИИ, 
ЧТО ЕГО МОЖНО ПРИМЕНЯТЬ ДЛЯ ЗАПИСИ МНОГИХ РАЗЛИЧНЫХ ПРОГРАММ, 
И ДЛЯ ВСЕХ ЭТИХ ПРОГРАММ НАМ ЖЕЛАТЕЛЬНО ЗНАТЬ СООТВЕТСТВУЮЩИЕ 
ИМ ПРЕОБРАЗОВАТЕЛИ ПРЕДИКАТОВ.

 кАЖДАЯ ТАКАЯ ПРОГРАММА ЗАДАЕТСЯ СВОИМ ТЕКСТОМ НА ХОРОШО 
ОПРЕДЕЛЕННОМ ЯЗЫКЕ ПРОГРАММИРОВАНИЯ, И ПОЭТОМУ ЕЕ ТЕКСТ 
ДОЛЖЕН СЛУЖИТЬ ДЛЯ НАС ОТПРАВНОЙ ТОЧКОЙ. нО ТЕПЕРЬ ПЕРЕД НАМИ 
НЕОЖИДАННО ОТКРЫВАЮТСЯ ДВА СОВЕРШЕННО РАЗЛИЧНЫХ НАЗНАЧЕНИЯ 
ТАКОГО ТЕКСТА ПРОГРАММЫ. с ОДНОЙ СТОРОНЫ, ТЕКСТ ПРОГРАММЫ 
ПРЕДНАЗНАЧЕН ДЛЯ \emph{МАШИННОЙ} ИНТЕРПРЕТАЦИИ, ЕСЛИ МЫ 
ХОТИМ, ЧТОБЫ ОНА МОГЛА ВЫПОЛНЯТЬСЯ АВТОМАТИЧЕСКИ, ЕСЛИ МЫ 
ХОТИМ, ЧТОБЫ ПО НЕЙ ДЛЯ НАС БЫЛ ПРОИЗВЕДЕН КАКОЙ-ЛИБО 
КОНКРЕТНЫЙ РАСЧЕТ. с ДРУГОЙ СТОРОНЫ, ЖЕЛАТЕЛЬНО, ЧТОБЫ ТЕКСТ 
ПРОГРАММЫ ГОВОРИЛ \emph{НАМ} О ТОМ, КАК СТРОИТЬ 
СООТВЕТСТВУЮЩИЙ ПРЕОБРАЗОВАТЕЛЬ ПРЕДИКАТОВ, КАК ПРОИЗВОДИТЬ 
ПРЕОБРАЗОВАНИЕ ПРЕДИКАТОВ, ЧТОБЫ ВЫВОДИТЬ ПРЕДУСЛОВИЕ 
$\wp(S, R)$ ДЛЯ ЛЮБОГО ДАННОГО ПОСТУСЛОВИЯ $R$, КОТОРОЕ НАС 
ПОЧЕМУ-ЛИБО ЗАИНТЕРЕСОВАЛО. эТО ЗАМЕЧАНИЕ ПОЗВОЛЯЕТ ПОНЯТЬ, 
ЧТО ПОДРАЗУМЕВАЕТСЯ ПОД "ХОРОШО ОПРЕДЕЛЕННЫМ ЯЗЫКОМ 
ПРОГРАММИРОВАНИЯ" С \emph{НАШЕЙ} ТОЧКИ ЗРЕНИЯ. кОГДА 
СЕМАНТИКА КОНКРЕТНОЙ КОНСТРУКЦИИ (ИЛИ ПРОГРАММЫ) ЗАДАЕТСЯ ЕЕ 
ПРЕОБРАЗОВАТЕЛЕМ ПРЕДИКАТОВ, МЫ РАССМАТРИВАЕМ СЕМАНТИЧЕСКУЮ 
ХАРАКТЕРИСТИКУ ЯЗЫКА ПРОГРАММИРОВАНИЯ КАК НАБОР ПРАВИЛ, 
КОТОРЫЕ ПОЗВОЛЯЮТ ЛЮБОЙ ПРОГРАММЕ, НАПИСАННОЙ ПА ЭТОМ ЯЗЫКЕ, 
ПОСТАВИТЬ В СООТВЕТСТВИЕ ПРЕОБРАЗОВАТЕЛЬ ПРЕДИКАТОВ. с ТАКОЙ 
ТОЧКИ ЗРЕНИЯ МЫ МОЖЕМ РАССМАТРИВАТЬ ПРОГРАММУ КАК "КОД" ДЛЯ 
СООТВЕТСТВУЮЩЕГО ПРЕОБРАЗОВАТЕЛЯ ПРЕДИКАТОВ.

пРИ ЖЕЛАНИИ МОЖНО ПОДОЙТИ К ПРОБЛЕМЕ ПРОЕКТИРОВАНИЯ ЯЗЫКА 
ПРОГРАММИРОВАНИЯ С ТАКОЙ ПОЗИЦИИ. мОЖНО РУКОВОДСТВОВАТЬСЯ 
(ДОВОЛЬНО ФОРМАЛЬНО) ТЕМ, ЧТО ПРАВИЛА ПОСТРОЕНИЯ 
ПРЕОБРАЗОВАТЕЛЕЙ ПРЕДИКАТОВ ДОЛЖНЫ БЫТЬ ТАКИМИ, ЧТОБЫ, 
ПРИМЕНЯЯ ИХ, НЕЛЬЗЯ БЫЛО ПОСТРОИТЬ НИЧЕГО ДРУГОГО, КРОМЕ КАК 
ПРЕОБРАЗОВАТЕЛЯ ПРЕДИКАТОВ, ОБЛАДАЮЩЕГО СВОЙСТВАМИ 1--4 ИЗ 
ПРЕДЫДУЩЕЙ ГЛАВЫ. в САМОМ ДЕЛЕ, ЕСЛИ ПРАВИЛА НЕ ДАЮТ ТАКОЙ 
ГАРАНТИИ, ТО ЭТО ОЗНАЧАЕТ, ЧТО ВЫ ДЕФОРМИРУЕТЕ ПРЕДИКАТЫ 
ТАКИМ ОБРАЗОМ, ЧТО ОНИ УЖЕ НЕ МОГУТ ИНТЕРПРЕТИРОВАТЬСЯ КАК 
ПОСТУСЛОВИЯ И СООТВЕТСТВУЮЩИЕ СЛАБЕЙШНЕ ПРЕДУСЛОВИЯ.

сРАЗУ НАПРАШИВАЮТСЯ ДВА ПРИМЕРА ВЕСЬМА ПРОСТЫХ 
ПРЕОБРАЗОВАТЕЛЕЙ ПРЕДИКАТОВ, КОТОРЫЕ ОБЛАДАЮТ ТРЕБУЕМЫМИ 
СВОЙСТВАМИ.

нАЧНЕМ С ТОЖДЕСТВЕННОГО ПРЕОБРАЗОВАНИЯ, Т. Е, С КОНСТРУКЦИИ 
$S$, ТАКОЙ, ЧТО ДЛЯ ЛЮБОГО ПОСТУСЛОВИЯ $R$ МЫ ИМЕЕМ 
$\wp(S, R)=R$. эТУ КОНСТРУКЦИЮ ЗНАЮТ И ЛЮБЯТ ВСЕ 
ПРОГРАММИСТЫ: ОНА ИЗВЕСТНА ИМ ПОД НАЗВАНИЕМ "ПУСТОЙ 
onepaТОР", И В СВОИХ ПРОГРАММАХ ОНИ ЧАСТО ИСПОЛЬЗУЮТ ЕЕ, 
ОСТАВЛЯЯ ПРОПУСК В ТОМ МЕСТЕ ТЕКСТА, ГДЕ СИНТАКСИЧЕСКИ 
ТРЕБУЕТСЯ КАКОЙ-ТО ОПЕРАТОР. эТО НЕ СЛИШКОМ ПОХВАЛЬНЫЙ ПРИЕМ 
(КОМПИЛЯТОР ДОЛЖЕН ЗНАТЬ, ЧТО ОН "ВИДИТ" ЭТОТ ОПЕРАТОР, НА 
ТОМ ОСНОВАНИИ, ЧТО  ОН НИЧЕГО НЕ  ВИДИТ, И ПОЭТОМУ МЫ ДАДИМ 
ЭТОЙ КОНСТРУКЦИИ НАИМЕНОВАНИЕ,  СКАЖЕМ, "\var{ПРОПУСТИТЬ}". 
иТАК, СЕМАНТИКА ОПЕРАТОРА "\var{ПРОПУСТИТЬ}"  ОПРЕДЕЛЯЕТСЯ 
СЛЕДУЮЩИМ ОБРАЗОМ: 
$$
 \wp(\var{ПРОПУСТИТЬ}, R)=R\qquad\hbox{ ДЛЯ ЛЮБОГО ПОСТУСЛОВИЯ $R$ }
$$

 (кАК И ВСЕ, Я БУДУ ПОЛЬЗОВАТЬСЯ ТЕРМИНОМ "ОПЕРАТОР", 
ПОСКОЛЬКУ ОН ПРОЧНО ВОШЕЛ В ЖАРГОН. кОГДА ЛЮДИ СООБРАЗИЛИ, 
ЧТО "КОМАНДА" МОГЛА БЫ ОКАЗАТЬСЯ БОЛЕЕ ПОДХОДЯЩИМ ТЕРМИНОМ, 
БЫЛО УЖЕ СЛИШКОМ  ПОЗДНО%
\note{  пО ТРАДИЦИИ МЫ ПЕРЕВОДИМ АНГЛИЙСКИЙ ТЕРМИН 
"statement"  (УТВЕРЖДЕНИЕ, ПРЕДЛОЖЕНИЕ) ТЕРМИНОМ "ОПЕРАТОР", 
ВВЕДЕННЫМ В ПРОГРАММИРОВАНИЕ а. а. лЯПУНОВЫМ, И ТАКИМ 
ОБРАЗОМ, РУССКИЙ ЧИТАТЕЛЬ ОКАЗЫВАЕТСЯ В БОЛЕЕ ВЫГОДНОМ 
ПОЛОЖЕНИИ, ЧЕМ АНГЛИЙСКИЙ.---{\it пРИМ. РЕД.} }.)

{\sl зАМЕЧАНue.} тЕМ, КТО СЧИТАЕТ РАСТОЧИТЕЛЬСТВОМ СИМВОЛОВ 
ВВЕДЕНИЕ ЯВНОГО ИМЕНИ "\var{ПРОПУСТИТЬ}" ДЛЯ ПУСТОГО 
ОПЕРАТОРА, КОГДА "ПУСТО" СТОЛЬ КРАСНОРЕЧИВО ВЫРАЖАЕТ ЕГО 
СЕМАНТИКУ, СЛЕДУЕТ ОСОЗНАТЬ, ЧТО ДЕСЯТИЧНАЯ СИСТЕМА 
СЧИСЛЕНИЯ ОКАЗАЛАСЬ ВОЗМОЖНОЙ ТОЛЬКО БЛАГОДАРЯ ВВЕДЕНИЮ 
СИМВОЛА "0" ДЛЯ ПОНЯТИЯ "НИЧТО". {\sl (кОНЕЦ ЗАМЕЧАНИЯ.)}

пРЕЖДЕ ЧЕМ ПРОДОЛЖИТЬ НАШИ РАССУЖДЕНИЯ, МНЕ ХОТЕЛОСЬ БЫ НЕ 
УПУСТИТЬ ВОЗМОЖНОСТЬ ОТМЕТИТЬ, ЧТО ТЕМ ВРЕМЕНЕМ МЫ УЖЕ 
ОПРЕДЕЛИЛИ НЕКИЙ ЯЗЫК ПРОГРАММИРОВАНИЯ. оСТАЕТСЯ ДОБАВИТЬ 
ТОЛЬКО ОДНО: ЭТО ОДНООПЕРАТОРПЫЙ ЯЗЫК, В КОТОРОМ МОЖНО 
ОПИСАТЬ ТОЛЬКО ОДНУ КОНСТРУКЦИЮ, ПРИЧЕМ ЕДИНСТВЕННОЕ, ЧТО 
СПОСОБНА СДЕЛАТЬ ДЛЯ НАС ДАННАЯ КОНСТРУКЦИЯ, ЭТО "ОСТАВИТЬ 
Вce, КАК ЕСТЬ" (ИЛИ "НИЧЕГО НЕ ДЕЛАТЬ", НО ТАКОЕ НЕГАТИВНОЕ 
УПОТРЕБЛЕНИЕ ЯЗЫКА ПРЕДСТАВЛЯЕТ ОПАСНОСТЬ, СМ. СЛЕДУЮЩИЙ 
АБЗАЦ).

дРУГОЙ ПРОСТОЙ ПРЕОБРАЗОВАТЕЛЬ ПРЕДИКАТОВ ПРИВОДИТ К 
ПОСТОЯННОМy СЛАБЕЙШЕМУ ПРЕДУСЛОВИЮ, КОТОРОЕ ВОВСЕ НЕ ЗАВИСИТ 
ОТ ПОСТУСЛОВИЯ $R$. мЫ ИМЕЕМ ДВА ПРЕДИКАТА-КОНСТАНТЫ, $T$ И 
$F$. кОНСТРУКЦИЯ $S$, ДЛЯ КОТОРОЙ  $\wp(S, R)=T$ ПРИ ВСЕХ 
$R$, НЕ МОЖЕТ СУЩЕСТВОВАТЬ, ПОТОМУ ЧТО ОНА НАРУШИЛА БЫ ЗАКОН 
ИСКЛЮЧЕННОГО ЧУДА. оДНАКО КОНСТРУКЦИЯ $S$, ДЛЯ КОТОРОЙ 
$\wp(S,R)=F$ ПРИ ВСЕХ $R$, ОБЛАДАЕТ ПРЕОБРАЗОВАТЕЛЕМ 
ПРЕДИКАТОВ, УДОВЛЕТВОРЯЮЩИМ ВСЕМ НЕОБХОДИМЫМ СВОЙСТВАМ. эТОМУ ОПЕРАТОРУ 
МЫ ТОЖЕ ПРИСВОИМ ИМЯ, НАЗОВЕМ ЕГО "\var{ОТКАЗАТЬ}". 
иТАК, СЕМАНТИКА ОПЕРАТОРА "\var{ОТКАЗАТЬ}" ЗАДАЕТСЯ СЛЕДУЮЩИМ ОБРАЗОМ:
$$
 \wp (\var{ОТКАЗАТЬ}, R) = F\qquad\hbox{ ДЛЯ ЛЮБОГО ПОСТУСЛОВИЯ R} 
$$
эТОТ ОПЕРАТОР НЕ МОЖЕТ ДАЖЕ "НИЧЕГО НЕ ДЕЛАТЬ" В СМЫСЛЕ 
"ОСТАВИТЬ ВСЕ, КАК ЕСТЬ"; ОН ВООБЩЕ НИ НА ЧТО НЕ СПОСОБЕН. 
еСЛИ МЫ ПОЛАГАЕМ $R=T$, Т. Е. НЕ НАКЛАДЫВАЕМ НА КОНЕЧНОЕ 
СОСТОЯНИЕ НИКАКИХ ДОПОЛНИТЕЛЬНЫХ ТРЕБОВАНИЙ, КРОМЕ САМОГО 
ФАКТА ЕГО СУЩЕСТВОВАНИЯ, ДАЖЕ ТОГДА НЕ НАЙДЕТСЯ 
СООТВЕТСТВУЮЩЕГО НАЧАЛЬНОГО СОСТОЯНИЯ. еСЛИ ЗАПУСТИТЬ 
КОНСТРУКЦИЮ ПО ИМЕНИ "\var{ОТКАЗАТЬ}", ОНА НЕ СМОЖЕТ ДОСТИЧЬ 
КОНЕЧНОГО СОСТОЯНИЯ: САМА ПОПЫТКА ЕЕ ЗАПУСКА ЯВЛЯЕТСЯ 
ГАРАНТИЕЙ НЕУДАЧИ. (нАС НЕ ДОЛЖНО ЗАНИМАТЬ ТЕПЕРЬ (А ТАКЖЕ И 
ВПОСЛЕДСТВИИ) ТО, ЧТО ПОЗДНЕЕ МЫ ВВЕДЕМ СТРУКТУРУ 
ОПЕРАТОРОВ, В КОТОРОЙ СОДЕРЖАТСЯ КАК ЧАСТНЫЕ СЛУЧАИ 
СЕМАНТИЧЕСКИЕ ЭКВИВАЛЕНТЫ ДЛЯ "\var{ПРОПУСТИТЬ}" И 
"\var{ОТКАЗАТЬ}".)

тЕПЕРЬ МЫ РАСПОЛАГАЕМ НЕКИМ (ВСЕ ЕЩЕ ВЕСЬМА ЗАЧАТОЧНЫМ) 
ДВУХОПЕРАТОРНЫМ ЯЗЫКОМ ПРОГРАММИРОВАНИЯ, В КОТОРОМ МОЖЕМ 
ОПИСАТЬ ДВЕ КОНСТРУКЦИИ; ОДНА ИЗ НИХ НИЧЕГО НЕ ДЕЛАЕТ, А 
ВТОРАЯ ВСЕГДА ТЕРПИТ НЕУДАЧУ. сО ВРЕМЕНИ ОПУБЛИКОВАНИЯ 
ЗНАМЕНИТОГО "сОБЩЕНИЯ ОБ АЛГОРИТМИЧЕСКОМ ЯЗЫКЕ алгол 60" В 
1960~Г. НИКАКОЙ УВАЖАЮЩИЙ СЕБЯ УЧЕНЫЙ, ЗАНИМАЮЩИЙСЯ 
ПРОГРАММИРОВАНИЕМ, НЕ ПОЗВОЛИТ СЕБЕ ОБОЙТИСЬ НА ЭТОМ ЭТАПЕ 
БЕЗ ФОРМАЛЬНОГО ОПРЕДЕЛЕНИЯ СИНТАКСИСА СТОЛЬ ДАЛЕКО 
ПРОДВИНУТОГО ЯЗЫКА ПРОГРАММИРОВАНИЯ В СИСТЕМЕ ОБОЗНАЧЕНИЙ, 
НАЗЫВАЕМОЙ "нфб" (СОКРАЩЕНИЕ ОТ "нОРМАЛЬНАЯ ФОРМА бЭКУСА"), 
А ИМЕННО:
$$  
\<ОПЕРАТОР>:: = \var{ПРОПУСТИТЬ} | \var{ОТКАЗАТЬ}
$$
(чИТАЕТСЯ ТАК: "эЛЕМЕНТ СИНТАКСИЧЕСКОЙ КАТЕГОРИИ, ИМЕНУЕМОЙ 
"ОПЕРАТОР" (ИМЕННО ЭТО ОБОЗНАЧАЮТ ЗАБАВНЫЕ СКОБКИ "$<$" И 
"$>$"), ОПРЕДЕЛЯЕТСЯ КАК (ЭТО ОБОЗНАЧАЕТ ЗНАК "$::=$") 
"\var{ПРОПУСТИТЬ}" ИЛИ (ЭТО ОБОЗНАЧАЕТ ВЕРТИКАЛЬНАЯ ЧЕРТА 
"$|$") "\var{ОТКАЗАТЬ}". кОЛОССАЛЬНО! нО НЕ БЕСПОКОЙТЕСЬ; 
БОЛЕЕ ВПЕЧАТЛЯЮЩИЕ ПРИМЕНЕНИЯ нфб В КАЧЕСТВЕ СПОСОБА ЗАПИСИ 
ПОСЛЕДУЮТ В НАДЛЕЖАЩЕЕ ВРЕМЯ.)

оДИН КЛАСС БЕЗУСЛОВНО БОЛЕЕ ИНТЕРЕСНЫХ ПРЕОБРАЗОВАТЕЛЕЙ 
ПРЕДИКАТОВ ОСНОВЫВАЕТСЯ НА ПОДСТАНОВКЕ, Т. Е. НА  ЗАМЕНЕ 
ВСЕХ ВХОЖДЕНИЙ НЕКОЕЙ ПЕРЕМЕННОЙ В ФОРМАЛЬНОМ ВЫРАЖЕНИИ ДЛЯ 
ПОСТУСЛОВИЯ $R$ НА (ОДНО И ТО ЖЕ) "ЧТО-НИБУДЬ ДРУГОЕ". еСЛИ 
В ПРЕДИКАТЕ $R$ ВСЕ ВХОЖДЕНИЯ ПЕРЕМЕННОЙ $x$ ЗАМЕНЯЮТСЯ 
НЕКОТОРЫМ ВЫРАЖЕНИЕМ $(E)$, ТО МЫ ОБОЗНАЧАЕМ РЕЗУЛЬТАТ ЭТОГО 
ПРЕОБРАЗОВАНИЯ ЧЕРЕЗ $R_{E\to X}$. тЕПЕРЬ МЫ МОЖЕМ 
РАССМОТРЕТЬ ДЛЯ ЗАДАННЫХ $x$ И $E$ ТАКУЮ КОНСТРУКЦИЮ, ЧТОБЫ 
ДЛЯ ЛЮБЫХ ПОСТУСЛОВИЙ $R$ ПОЛУЧАЛОСЬ $\wp(S, R) =R_{E\to X}$; 
ЗДЕСЬ $x$ --- "КООРДИНАТНАЯ ПЕРЕМЕННАЯ" НАШЕГО ПРОСТРАНСТВА 
СОСТОЯНИЙ, А $E$ ---  ВЫРАЖЕНИЕ СООТВЕТСТВУЮЩЕГО ТИПА.

{\sl зАМЕЧАНИЕ.} тАКОЕ ПРЕОБРАЗОВАНИЕ ПУТЕМ ПОДСТАНОВКИ 
УДОВЛЕТВОРЯЕТ СВОЙСТВАМ 1--4 ИЗ ПРЕДЫДУЩЕЙ ГЛАВЫ. мЫ НЕ 
СТАНЕМ ПЫТАТЬСЯ ДЕМОНСТРИРОВАТЬ ЭТО И ПРЕДОСТАВИМ САМОМУ  
ЧИТАТЕЛЮ РЕШАТЬ В ЗАВИСИМОСТИ ОТ СВОЕГО ВКУСА, БУДЕТ ЛИ ОН 
ОТНОСИТЬСЯ К ЭТОМУ КАК К ТРИВИАЛЬНОМУ ИЛИ ЖЕ КАК К ГЛУБОКОМУ 
МАТЕМАТИЧЕСКОМУ РЕЗУЛЬТАТУ. {\sl(кОНЕЦ ЗАМЕЧАНИЯ.)}

тАКИМ СПОСОБОМ ВВОДИТСЯ ЦЕЛЫЙ КЛАСС ПРЕОБРАЗОВАТЕЛЕЙ 
ПРЕДИКАТОВ, ЦЕЛЫЙ КЛАСС КОНСТРУКЦИЙ. оНИ ОБОЗНАЧАЮТСЯ 
ОПЕРАТОРОМ, КОТОРЫЙ НАЗЫВАЕТСЯ "ОПЕРАТОР ПРИСВАИВАНИЯ"; 
ТАКОЙ ОПЕРАТОР ДОЛЖЕН ОПРЕДЕЛЯТЬ ТРИ ВЕЩИ:

1) НАИМЕНОВАНИЕ ПЕРЕМЕННОЙ, ПОДЛЕЖАЩЕЙ ЗАМЕНЕ; 

2) ТОТ ФАКТ, ЧТО ПРАВИЛО, СООТВЕТСТВУЮЩЕЕ ПРЕОБРАЗОВАНИЮ 
ПРЕДИКАТОВ, ЕСТЬ ПОДСТАНОВКА;

 3) ВЫРАЖЕНИЕ, КОТОРЫМ ДОЛЖНО ЗАМЕНЯТЬСЯ ВСЯКОЕ ВХОЖДЕНИЕ 
ЭТОЙ ПЕРЕМЕННОЙ В ПОСТУСЛОВИИ.

еСЛИ ПЕРЕМЕННАЯ $x$ ДОЛЖНА БЫТЬ ЗАМЕНЕНА ВЫРАЖЕНИЕМ $(E)$ ТО 
ОБЫЧНАЯ ЗАПИСЬ ТАКОГО ОПЕРАТОРА ВЫГЛЯДИТ СЛЕДУЮЩИМ
ОБРАЗОМ: 
$$  
x:=E 
$$ 
(ГДЕ ТАК НАЗЫВАЕМЫЙ ЗНАК ПРИСВАИВАНИЯ ":=" СЛЕДУЕТ ЧИТАТЬ КАК 
"СТАНОВИТСЯ"). 

сКАЗАННОЕ МОЖНО СУММИРОВАТЬ, ОПРЕДЕЛИВ
$$ 
\wp("x:=E", R) =R_{E\to X} \qquad\hbox{дЛЯ ЛЮБОГО ПОСТУСЛОВИЯ $R$}
$$

пРИ ЖЕЛАНИИ ЭТО МОЖНО РАССМАТРИВАТЬ КАК СЕМАНТИЧЕСКОЕ 
ОПРЕДЕЛЕНИЕ ОПЕРАТОРА ПРИСВАИВАНИЯ ДЛЯ ЛЮБОЙ КООРДИНАТНОЙ 
ПЕРЕМЕННОЙ $x$ И ЛЮБОГО ВЫРАЖЕНИЯ $E$ СООТВЕТСТВУЮЩЕГО ТИПА. 
нАСЛАЖДАЯСЬ, ПО НАШЕМУ ОБЫКНОВЕНИЮ, УПОТРЕБЛЕНИЕМ нфб, МЫ 
МОЖЕМ РАСШИРИТЬ НАШ ФОРМАЛЬНЫЙ СИНТАКСИС, ЧТОБЫ ОН ЧИТАЛСЯ 
ТАК:
$$ 
\eqalign{
&\<ОПЕРАТОР> ::=\var{ПРОПУСТИТЬ}|\var{ОТКАЗАТЬ}|\<ОПЕРАТОР ПРИСВАИВАНИЯ> 
\cr
&\<ОПЕРАТОР ПРИСВАИВАНИЯ> ::=\<ПЕРЕМЕННАЯ> := \<ВЫРАЖЕНИЕ> \cr
}
$$
ПРИЧЕМ ПОСЛЕДНЮЮ СТРОЧКУ СЛЕДУЕТ ЧИТАТЬ ТАК: "эЛЕМЕНТ 
СИНТАКСИЧЕСКОЙ КАТЕГОРИИ, ИМЕНУЕМОЙ "ОПЕРАТОР ПРИСВАИВАНИЯ", 
ОПРЕДЕЛЯЕТСЯ КАК ЭЛЕМЕНТ СИНТАКСИЧЕСКОЙ КАТЕГОРИИ, ИМЕНУЕМОЙ 
"ПЕРЕМЕННАЯ", ЗА КОТОРЫМ СЛЕДУЮТ СНАЧАЛА ЗНАК ПРИСВАИВАНИЯ 
"$:=$", А ЗАТЕМ ЭЛЕМЕНТ СИНТАКСИЧЕСКОЙ КАТЕГОРИИ, ИМЕНУЕМОЙ 
"ВЫРАЖЕНИЕ".
пЕРЕД ТЕМ КАК ПРОДОЛЖИТЬ РАССУЖДЕНИЕ, ПРЕДСТАВЛЯЕТСЯ РАЗУМНЫМ 
УБЕДИТЬСЯ В ТОМ, ЧТО ЭТИМ ФОРМАЛЬНЫМ ОПИСАНИЕМ ОПЕРАТОРА 
ПРИСВАИВАНИЯ В САМОМ ДЕЛЕ ОХВАТЫВАЕТСЯ НАШЕ ИНТУИТИВНОЕ 
ПОНИМАНИЕ ОПЕРАТОРА ПРИСВАИВАНИЯ --- ЕСЛИ ОНО У НАС ЕСТЬ! 
рАССМОТРИМ ПРОСТРАНСТВО СОСТОЯНИЙ С ДВУМЯ ЦЕЛЫМИ 
КООРДИНАТНЫМИ ПЕРЕМЕННЫМИ "$a$" И "$b$". тОГДА
$$ 
\wp("a:=7", a=7)= \{7=7\} 
$$
И, ПОСКОЛЬКУ ЛОГИЧЕСКОЕ ВЫРАЖЕНИЕ В ПРАВОЙ ЧАСТИ ЯВЛЯЕТСЯ 
ИСТИНОЙ ДЛЯ ВСЕХ ЗНАЧЕНИЙ $a$ И $b$, Т. Е. ДЛЯ ВСЕХ ТОЧЕК 
ПРОСТРАНСТВА СОСТОЯНИЙ, МЫ МОЖЕМ УПРОЩЕННО ЗАПИСАТЬ ЭТО ТАК:
$$ 
\wp("a:=7", a=7)=T 
$$
иНАЧЕ ГОВОРЯ, ВСЯКОЕ НАЧАЛЬНОЕ СОСТОЯНИЕ ГАРАНТИРУЕТ, ЧТО 
ПРИСВАИВАНИЕ "$a:=7$" ОБЕСПЕЧИТ ИСТИННОСТЬ "$a=7$". аНАЛОГИЧНО
$$ 
\wp("a:=7", a=6) = \{7=6\} 
$$
И, ПОСКОЛЬКУ ЭТО ЛОГИЧЕСКОЕ ВЫРАЖЕНИЕ ЯВЛЯЕТСЯ ЛОЖЬЮ ДЛЯ ВСЕХ 
ЗНАЧЕНИЙ $a$ И $b$, МЫ ПОЛУЧАЕМ
$$
 \wp( "a:=7", a=6)=F 
$$

эТО ОЗНАЧАЕТ, ЧТО НЕ СУЩЕСТВУЕТ НИКАКОГО НАЧАЛЬНОГО СОСТОЯНИЯ, ДЛЯ 
КОТОРОГО МЫ МОГЛИ БЫ ГАРАНТИРОВАТЬ, ЧТО ПРИСВАИВАНИЕ "$a:=7$" 
ОБЕСПЕЧИТ ИСТИННОСТЬ "$a=6$". (эТО НАХОДИТСЯ В СООВЕТСТВИИ С НАШИМ 
ПРЕДЫДУЩИМ РЕЗУЛЬТАТОМ, ЧТО ВСЕ НАЧАЛЬНЫЕ СОСТОЯНИЯ ОБЕСПЕЧАТ 
КОНЕЧНУЮ ИСТИННОСТЬ "$a=7$", А  СЛЕДОВАТЕЛЬНО, КОНЕЧНУЮ ЛОЖНОСТЬ 
"$a\not=7$".) дАЛЕЕ,
$$ 
\wp ("a:=7", b=b0)=\{b=b0\}
$$
Т. Е. ЕСЛИ МЫ ХОТИМ ГАРАНТИРОВАТЬ, ЧТО ПОСЛЕ ПРИСВАИВАНИЯ "$a:=7$" 
ПЕРЕМЕННАЯ b ИМЕЕТ НЕКОТОРОЕ ЗНАЧЕНИЕ $b0$, ТО НУЖНО, ЧТОБЫ $b$ ИМЕЛА 
ЭТО ЗНАЧЕНИЕ ЕЩЕ В НАЧАЛЬНОМ СОСТОЯНИИ. дРУГИМИ СЛОВАМИ, ВСЕ 
ПЕРЕМЕННЫЕ, ЗА ИСКЛЮЧЕНИЕМ "$a$", НЕ ИЗМЕНЯЮТСЯ, ОНИ СОХРАНЯЮТ ТЕ 
ЗНАЧЕНИЯ, КОТОРЫЕ ИМЕЛИ РАНЬШЕ; ПРИСВАИВАННЕ "$a:=7$" ПЕРЕМЕЩАЕТ В 
ПРОСТРАНСТВЕ СОСТОЯНИЙ ТОЧКУ, СООТВЕТСТВУЮЩУЮ ТЕКУЩЕМУ СОСТОЯНИЮ 
СИСТЕМЫ, ПАРАЛЛЕЛЬНО ОСИ $a$  ТАК, ЧТО ОБЕСПЕЧИВАЕТСЯ КОНЕЧНОЕ 
ВЫПОЛНЕНИЕ РАВЕНСТВА "$a=7$".

вМЕСТО ТОГО ЧТОБЫ БРАТЬ В КАЧЕСТВЕ ВЫРАЖЕНИЯ $E$ КОНСТАНТУ,  МЫ 
МОГЛИ БЫ ВЗЯТЬ КАКУЮ-ТО ФУНКЦИЮ ОТ НАЧАЛЬНОГО  СОСТОЯНИЯ. эТО 
ИЛЛЮСТРИРУЕТСЯ СЛЕДУЮЩИМИ ПРИМЕРАМИ:
$$ 
\eqalign{
&\wp("a:=2*b+1", a=13)=\{2*b+1=13\}=\{b=6\} \cr
&\wp ("a: =a +1", a>10)=\{a+1>10\}=\{a>9\} \cr
&\wp("a:=a-b", a>b) = \{a-b>b\}=\{a>2*b\} \cr
}
$$

вОЗНИКАЕТ НЕБОЛЬШОЕ ОСЛОЖНЕНИЕ, ЕСЛИ МЫ РАЗРЕШАЕМ ВЫРАЖЕНИЮ $E$ БЫТЬ 
ЧАСТИЧНО ОПРЕДЕЛЕННОЙ ФУНКЦИЕЙ НАЧАЧАЛЬНОГО СОСТОЯНИЯ, Т. Е. ТАКОЙ 
ФУНКЦИЕЙ, ПОПЫТКА ВЫЧИСЛЕНИЯ КОТОРОЙ ДЛЯ НАЧАЛЬНОГО СОСТОЯНИЯ, НЕ 
ВХОДЯЩЕГО В ОБЛАСТЬ ОПРЕДЕЛЕНИЯ, НЕ ПРИВЕДЕТ К ПРАВИЛЬНО ЗАВЕРШАЕМОЙ 
РАБОТЕ. еСЛИ МЫ ХОТИМ ПРЕДУСМОТРЕТЬ И ЭТУ СИТУАЦИЮ, ТО НАМ НУЖНО 
УТОЧНИТЬ НАШЕ ОПРЕДЕЛЕНИЕ СЕМАНТИКИ ОПЕРАТОРА ПРИСВАИВАИВАНИЯ И 
ЗАПИСАТЬ
$$ 
\wp ("x:=E", R) = \{D(E) \cand R_{E \to X}\} 
$$
зДЕСЬ ПРЕДИКАТ $D(E)$ ОЗНАЧАЕТ "В ОБЛАСТИ ОПРЕДЕЛЕНИЯ $E$"; 
ЛОГИЧЕСКОЕ ВЫРАЖЕНИЕ "$B1 \cand B2$" (ТАК НАЗЫВАЕМАЯ "УСЛОВНАЯ 
КОНRЮНКЦИЯ") ИМЕЕТ ТО ЖЕ ЗНАЧЕНИЕ, ЧТО И "$B1\and B2$", ЕСЛИ ОБА 
ОПЕРАНДА ОПРЕДЕЛЕНЫ, НО ПОМИМО ЭТОГО ПО ОПРЕДЕЛЕНИЮ ОНО ИМЕЕТ 
ЗНАЧЕНИЕ "ЛОЖЬ", ЕСЛИ $B1$ ЯВЛЯЕТСЯ "ЛОЖЬЮ"; ПОСЛЕДНЕЕ  СПРАВЕДЛИВО 
ВНЕ ЗАВИСИМОСТИ ОТ ТОГО, ОПРЕДЕЛЕН ЛИ ОПЕРАНД $B2$. оБЫЧНО УСЛОВИЕ 
$D(E)$ НЕ ПОДЧЕРКИВАЕТСЯ ЯВНО ЛИБО ПОТОМУ, ЧТО ОНО=$T$, ЛИБО ПОТОМУ, 
ЧТО МЫ ИСХОДИМ ИЗ ПРЕДПОЛОЖЕНИЯ, ЧТО ОПЕРАТОР ПРИСВАИВАНИЯ НИКОГДА 
НЕ БУДЕТ ЗАПУЩЕН В НАЧАЛЬНЫХ СОСТОЯНИЯХ, НЕ  ПРИНАДЛЕЖАЩИХ ОБЛАСТИ 
ОПРЕДЕЛЕНИЯ ВЫРАЖЕНИЯ $E$.

еСТЕСТВЕННЫМ ОБОБЩЕНИЕМ ОПЕРАТОРА ПРИСВАИВАНИЯ ЯВЛЯЕТСЯ 
ИЗЛЮБЛЕННОЕ НЕКОТОРЫМИ ПРОГРАММИСТАМИ ТАК НАЗЫВАЕМОЕ 
"ОДНОВРЕМЕННОЕ ПРИСВАИВАНИЕ". в ЭТОМ СЛУЧАЕ ВОЗМОЖНА 
ОДНОВРЕМЕННАЯ ЗАМЕНА ДЛЯ НЕСКОЛЬКИХ \emph{РАЗЛИЧНЫХ}ПЕРЕМЕННЫХ. 
оПЕРАТОР ОДНОВРЕМЕННОГО ПРИСВАИВАНИЯ 
ОБОЗНАЧАЕТСЯ СПИСКОМ РАЗЛИЧНЫХ ПЕРЕМЕННЫХ), ПОДЛЕЖАЩИХ 
ЗАМЕНЕ (ЭТИ ПЕРЕМЕННЫЕ РАЗДЕЛЯЮТСЯ ЗАПЯТЫМИ), СЛЕВА ОТ ЗНАКА 
ПРИСВАИВАНИЯ И СТОЛЬ ЖЕ ПРОТЯЖЕННЫМ СПИСКОМ ВЫРАЖЕНИЙ (ТАКЖЕ 
РАЗДЕЛЯЕМЫХ ЗАПЯТЫМИ) СПРАВА ОТ ЗНАКА ПРИСВАИВАНИЯ. иТАК, 
РАЗРЕШАЕТСЯ ПИСАТЬ
$$ 
x1,x2:=E1, E2 \qquad x1,x2, x3:=E1, E2, Eз 
$$
зАМЕТИМ, ЧТО $i$-Я  ПЕРЕМЕННАЯ ИЗ СПИСКА ЛЕВОЙ ЧАСТИ ДОЛЖНА 
БЫТЬ ЗАМЕНЕНА НА $i$-e ВЫРАЖЕНИЕ ИЗ СПИСКА ПРАВОЙ ЧАСТИ, ТАК 
ЧТО, НАПРИМЕР, ПРИ ЗАДАННЫХ $x1$, $x2$, $E1$ И $E2$ ОПЕРАТОР
$$ 
x1, x2:=E1,E2 
$$ 
СЕМАНТИЧЕСКИ ЭКВИВАЛЕНТЕН ОПЕРАТОРУ 
$$ 
x2,x1:=E2,E1 
$$
оДНОВРЕМЕННОЕ ПРИСВАИВАНИЕ ПОЗВОЛЯЕТ НАМ ПРЕДПИСАТЬ, ЧТОБЫ 
ДВЕ ПЕРЕМЕННЫЕ $x$ И $y$ ОБМЕНЯЛИСЬ СВОИМИ ЗНАЧЕНИЯМИ С 
ПОМОЩЬЮ ОПЕРАТОРА
$$ 
x, y:=y, x 
$$
в ИНОЙ ЗАПИСИ ЭТА ОПЕРАЦИЯ ВЫГЛЯДЕЛА БЫ БОЛЕЕ ГРОМОЗДКОЙ. 
лЕГКОСТЬ РЕАЛИЗАЦИИ ОДНОВРЕМЕННОГО ПРИСВАИВАНИЯ И 
ВОЗМОЖНОСТЬ С ЕГО ПОМОЩЬЮ ИЗБЕГАТЬ НЕКОТОРОЙ ИЗБЫТОЧНОСТИ 
ЗАПИСИ ЯВЛЯЮТСЯ ПРИЧИНАМИ ПОПУЛЯРНОСТИ ТАКИХ ОПЕРАТОРОВ. 
зАМЕТИМ, ЧТО ЕСЛИ СПИСКИ СТАНОВЯТСЯ СЛИШКОМ ДЛИННЫМИ, ТО 
ПОЛУЧАЕМАЯ ПРОГРАММА СТАНОВИТСЯ ВЕСЬМА НЕУДОБОЧИТАЕМОЙ. 

иСТИННЫЙ ЛЮБИТЕЛЬ нфб РАСШИРИТ СВОЙ СИНТАКСИС, ОБЕСПЕЧИВ ДВЕ 
РАЗЛИЧНЫЕ ФОРМЫ ДЛЯ ОПЕРАТОРА ПРИСВАИВАНИЯ, СЛЕДУЮЩИМ 
ОБРАЗОМ:
$$ 
\eqalign{
\<ОПЕРАТОР ПРИСВАИВАНИЯ>::=&\<ПЕРЕМЕННАЯ>:=\<ВЫРАЖЕНИЕ> | \cr
 &\<ПЕРЕМЕННАЯ>,  \<ОПЕРАТОР ПРИСВАИВАНИЯ>, \<ВЫРАЖЕНИЕ> \cr
}
$$
эТО ТАК НАЗЫВАЕМОЕ "РЕКУРСИВНОЕ ОПРЕДЕЛЕНИЕ", ПОСКОЛЬКУ ОДНА 
ИЗ АЛЬТЕРНАТИВНЫХ ФОРМ ДЛЯ СИНТАКСИЧЕСКОЙ ЕДИНИЦЫ, ИМЕНУЕМОЙ 
"ОПЕРАТОР ПРИСВАИВАНИЯ" (А ИМЕННО ВТОРАЯ ФОРМА), СОДЕРЖИТ В 
КАЧЕСТВЕ ОДНОГО ИЗ СВОИХ КОМПОНЕНТОВ СНОВА
ЭТУ ЖЕ СИНТАКСИЧЕСКУЮ ЕДИНИЦУ, ИМЕНУЕМУЮ "ОПЕРАТОР ПРИСВАИВАНИЯ", 
Т. Е. ТУ САМУЮ СИНТАКСИЧЕСКУЮ ЕДИНИЦУ, КОТОРУЮ МЫ ОПРЕДЕЛЯЕМ! нА 
ПЕРВЫЙ ВЗГЛЯД ТАКОЕ ЦИКЛИЧЕСКОЕ ОПРЕДЕЛЕНИЕ ВЫГЛЯДИТ УЖАСАЮЩЕ, НО 
ПОСЛЕ БОЛЕЕ ВНИМАТЕЛЬНОГО ИЗУЧЕНИЯ МЫ МОЖЕМ УБЕДИТЬСЯ В ТОМ, ЧТО, 
ПО КРАЙНЕЙ МЕРЕ С СИНТАКСИЧЕСКОЙ ТОЧКИ ЗРЕНИЯ, В ЭТОМ НЕТ НИЧЕГО 
НЕПРАВИЛЬНОГО. нАПРИМЕР, ПОСКОЛЬКУ, СОГЛАСНО ПЕРВОЙ АЛЬТЕРНАТИВЕ,
$$ 
x2:=E1 
$$
ЯВЛЯЕТСЯ ПРИМЕРОМ ОПЕРАТОРА ПРИСВАИВАНИЯ, ТО ФОРМУЛА
$$ 
x1,x2:=E1, E2 
$$
ДОПУСКАЕТ РАЗБОР ВИДА 
$$ x1, \<ОПЕРАТОР ПРИСВАИВАНИЯ>, E2 
$$
А СЛЕДОВАТЕЛЬНО, СОГЛАСНО ВТОРОЙ АЛЬТЕРНАТИВЕ, ТАКЖЕ ЯВЛЯЕТСЯ 
ОПЕРАТОРОМ ПРИСВАИВАНИЯ. оДНАКО С СЕМАНТИЧЕСКОЙ ТОЧКИ ЗРЕНИЯ ЭТО 
ОТВРАТИТЕЛЬНО, ПОТОМУ ЧТО ПОЛУЧАЕТСЯ, ЧТО $E2$ АССОЦИИРУЕТСЯ С 
$x1$ ВМЕСТО ТОГО, ЧТОБЫ АССОЦИИРОВАТЬСЯ С $x2$.

в СРАВНЕНИИ С ДВУХОПЕРАТОРНЫМ ЯЗЫКОМ, СОДЕРЖАЩИМ ТОЛЬКО 
"ПРОПУСТИТЬ" И "ОТКАЗАТЬ", НАШ ЯЗЫК С ОПЕРАТОРОМ ПРИСВАИВАНИЯ 
ВЫГЛЯДИТ ЗНАЧИТЕЛЬНО БОЛЕЕ БОГАТЫМ: УЖЕ НЕТ КАКОЙ БЫ ТО НИ БЫЛО 
ВЕРХНЕЙ ГРАНИЦЫ ДЛЯ ЧИСЛА РАЗЛИЧНЫХ ПРИМЕРОВ СИНТАКСИЧЕСКОЙ 
ЕДИНИЦЫ "ОПЕРАТОР ПРИСВАИВАНИЯ". нО ОН ВСЕ ЖЕ ЯВНО НЕДОСТАТОЧЕН 
ДЛЯ НАШИХ ЦЕЛЕЙ; НАМ НУЖНА ВОЗМОЖНОСТЬ СТРОИТЬ БОЛЕЕ ИЗОЩРЕННЫЕ 
ПРОГРАММЫ, БОЛЕЕ СЛОЖНЫЕ КОНСТРУКЦИИ. дЛЯ ПОСТРОЕНИЯ ПОТЕНЦИАЛЬНО 
СЛОЖНЫХ КОНСТРУКЦИЙ МЫ ПОЛЬЗУЕМСЯ СХЕМОЙ, КОТОРУЮ МОЖНО 
РЕКУРСИВНО ОПИСАТЬ ТАК:
$$
\displaylines{
\<КОНСТРУКЦИЯ>::=\<ПРОСТАЯ КОНСТРУКЦИЯ>| \hfill\cr
\hfill\<ПРАВИЛЬНАЯ КОМПОЗИЦИЯ ЗАПИСЕЙ ВИДА: {\<КОНСТРУКЦИЯ>} >\cr
}
$$ 

дЛЯ ТОГО ЧТОБЫ ЭТА СХЕМА МОГЛА ПРИНЕСТИ ХОТЬ КАКУЮ-НИБУДЬ ПОЛЬЗУ, 
НЕОБХОДИМО ВЫПОЛНЕНИЕ ДВУХ УСЛОВИЙ: В НАШЕМ РАСПОРЯЖЕНИИ ДОЛЖНЫ 
ИМЕТЬСЯ "ПРОСТЫЕ КОНСТРУКЦИИ", ЧТОБЫ БЫЛО С ЧЕГО НАЧАТЬ, И, КРОМЕ 
ТОГО, МЫ ДОЛЖНЫ ЗНАТЬ, КАК СТРОИТЬ "ПРАВИЛЬНЫЕ КОМПОЗИЦИИ". 
вВЕДЕННЫЕ РАНЬШЕ ОПЕРАТОРЫ МОЖНО ВЗЯТЬ В КАЧЕСТВЕ ПРОСТЫХ 
КОНСТРУКЦИЙ; ОСТАВШАЯСЯ ЧАСТЬ ЭТОЙ ГЛАВЫ ПОСВЯЩЕНА ИМЕННО 
ПРОЦЕССУ ПРАВИЛЬНОЙ КОМПОЗИЦИИ НЕКОЕЙ НОВОЙ КОНСТРУКЦИИ ИЗ 
ЗАДАННЫХ КОНСТРУКЦИЙ. нОВАЯ КОНСТРУКЦИЯ В СВОЮ ОЧЕРЕДЬ МОЖЕТ 
ВЫСТУПАТЬ В РОЛИ ЧАСТИ ЕЩЕ БОЛЕЕ СЛОЖНОГО  СОСТАВНОГО ОБRЕКТА. 

пОСЛЕ ТОГО, КАК ОБRЕКТ БЫЛ ОБРАЗОВАН КОМПОЗИЦИЕЙ ЧАСТЕЙ, МЫ МОЖЕМ 
РАССМАТРИВАТЬ ЕГО ДВУМЯ СПОСОБАМИ. вО-ПЕРВЫХ, МЫ
МОЖЕМ СЧИТАТЬ ЕГО "НЕРАЗДЕЛЬНЫМ ЦЕЛЫМ", ВОСПРИНИМАЯ ЕГО СВОЙСТВА 
В БОЛЬШЕЙ ИЛИ МЕНЬШЕЙ СТЕПЕНИ КАК МАГИЧЕСКИЕ (ИЛИ ПРИНИМАЯ ИХ НА 
ВЕРУ ИЛИ КАК ПОСТУЛАТЫ). пРИ ТАКОМ ПОДХОДЕ СУЩЕСТВЕННЫ ТОЛЬКО 
СВОЙСТВА КОНСТРУКЦИИ; НЕ ИМЕЕТ НИКАКoro ЗНАЧЕНИЯ, КАКИМ ОБРАЗОМ 
ОНА ОБРАЗОВАНА ИЗ СВОИХ ЧАСТЕЙ. с ТАКОЙ ТОЧКИ ЗРЕНИЯ ЛЮБЫЕ ДВЕ 
КОНСТРУКЦИИ, ОБЛАДАЮЩИЕ ОДИНАКОВЫМИ СВОЙСТВАМИ, ЭКВИВАЛЕНТНЫ. иЛИ 
ЖЕ МЫ РАССМАТРИВАЕМ КОНСТРУКЦИЮ КАК "СОСТАВНОЙ ОБRЕКТ", 
ОБРАЗОВАННЫЙ ТАК, ЧТО МЫ МОЖЕМ ПОНЯТЬ, ПОЧЕМУ ОНА ОБЛАДАЕТ 
ОБRЯВЛЕННЫМИ СВОЙСТВАМИ. пРИ ЭТОМ МЫ ВОСПРИНИМАЕМ ЧАСТИ КАК 
"МАЛЫЕ" НЕРАЗДЕЛЬНЫЕ ЦЕЛЫЕ, ДЛЯ КОТОРЫХ ПРИНИМАЮТСЯ В РАСЧЕТ 
ТОЛЬКО ИХ ОБЩИЕ СВОЙСТВА. вТОРОЙ ПОДХОД ВНОСИТ ЯСНОСТЬ В ТО, ЧТО 
МЫ ПОНИМАЕМ ПОД "КОМПОЗИЦИЕЙ". кОМПОЗИЦИЯ ДОЛЖНА  ОПРЕДЕЛЯТЬ, КАК 
СВОЙСТВА ЦЕЛОГО СЛЕДУЮТ ИЗ СВОЙСТВ ЧАСТЕЙ.

пОСЛЕ ЭТИХ ОБЩИХ ЗАМЕЧАНИЙ ВЕРНЕМСЯ К НАШИМ КОНКРЕТНЫМ 
КОНСТРУКЦИЯМ, СВОЙСТВА КОТОРЫХ, КАК МЫ СЧИТАЕМ, ВЫРАЖАЮТСЯ ИХ 
ПРЕОБРАЗОВАТЕЛЯМИ ПРЕДИКАТОВ. тОЧНЕЕ ГОВОРЯ, ЕСЛИ ЗАДАНЫ ДВЕ 
КОНСТРУКЦИИ $S1$ И $S2$, ЧЬИ ПРЕОБРАЗОВАТЕЛИ ПРЕДИКАТОВ ИЗВЕСТНЫ, 
МОЖЕМ ЛИ МЫ ПРЕДСТАВИТЬ СЕБЕ ПРАВИЛО ВЫВОДА НОВОГО 
ПРЕОБРАЗОВАТЕЛЯ ПРЕДИКАТОВ ИЗ ДВУХ ЗАДАННЫХ? еСЛИ ДА, ТО МЫ МОЖЕМ 
СЧИТАТЬ РЕЗУЛЬТИРУЮЩИЙ ПРЕОБРАЗОВАТЕЛЬ ПРЕДИКАТОВ ОПИСАНИЕМ 
СВОЙСТВ СОСТАВНОГО ОБRЕКТА, ПОСТРОЕННОГО СПЕЦИАЛЬНЫМ ОБРАЗОМ ИЗ 
ЧАСТЕЙ $S1$ И $S2$. 

оДНИМ ИЗ ПРОСТЕЙШИХ СПОСОБОВ ПОЛУЧЕНИЯ НОВОЙ ФУНКЦИИ ИЗ ДВУХ 
ЗАДАННЫХ ЯВЛЯЕТСЯ ТАК НАЗЫВАЕМАЯ "ФУНКЦИОНАЛЬНАЯ КОМПОЗИЦИЯ", Т. 
Е. ПРЕДОСТАВЛЕНИЕ ЗНАЧЕНИЯ ОДНОЙ ФУНКЦИИ В КАЧЕСТВЕ АРГУМЕНТА ДЛЯ 
ДРУГОЙ. сОСТАВНОЙ ОБRЕКТ, СООТВЕТСТВУЮЩИЙ ТАКОМУ ПРЕОБРАЗОВАТЕЛЮ 
ПРЕДИКАТОВ, ПРИНЯТО ОБОЗНАЧАТЬ ЧЕРЕЗ "S1; S2", И МЫ ОПРЕДЕЛЯЕМ
$$ 
\wp("S1; S2", R) =\wp(S1, \wp(S2, R)) 
$$
ЧТО ПРИ ЖЕЛАНИИ МОЖНО РАССМАТРИВАТЬ КАК СЕМАНТИЧЕСКОЕ ОПРЕДЕЛЕНИЕ 
ТОЧКИ С ЗАПЯТОЙ.

{\sl зАМЕЧАНue.} иЗ ТОГО ФАКТА, ЧТО ПРЕОБРАЗОВАТЕЛИ ПРЕДИКАТОВ 
ДЛЯ $S1$ И $S2$ ОБЛАДАЮТ СВОЙСТВАМИ 1--4 ИЗ ПРЕДЫДУЩЕЙ ГЛАВЫ, 
МОЖНО ЗАКЛЮЧИТЬ, ЧТО И ОПРЕДЕЛЕННЫЙ ВЫШЕ ПРЕОБРАЗОВАТЕЛЬ 
ПРЕДИКАТОВ ДЛЯ "$S1; S2$" ТАКЖЕ ОБЛАДАЕТ ЭТИМИ ЧЕТЫРЬМЯ 
СВОЙСТВАМИ. нАПРИМЕР, ПОСКОЛЬКУ ДЛЯ $S1$ И $S2$ СПРАВЕДЛИВ ЗАКОН 
ИСКЛЮЧЕННОГО ЧУДА:
$$ 
\wp(S1, F)=F\quad\hbox{ И }\quad\wp(S2, F)=F 
$$
МЫ ЗАКЛЮЧАЕМ, ПОДСТАВЛЯЯ $F$ ВМЕСТО $R$ В ВЕРХНЕЕ ОПРЕДЕЛЕНИЕ ЧТО 
$$
\eqalign{
 \wp("S1; S2", F)&=\wp(S1, \wp(S2, F))\cr
&=\wp(S1, F) \cr
&=F \cr
}
$$

чИТАТЕЛЮ ПРЕДОСТАВЛЯЕТСЯ В КАЧЕСТВЕ УПРАЖНЕНИЯ ПРОВЕРИТЬ, ЧТО 
ОСТАЛЬНЫЕ ТРИ СВОЙСТВА ТОЖЕ СОХРАНЯЮТСЯ. {\sl (кОНЕЦ ЗАМЕЧАНИЯ.)}

пЕРЕД ТЕМ КАК ПРОДОЛЖИТЬ НАШИ РАССУЖДЕНИЯ, УБЕДИМСЯ  В ТОМ, ЧТО 
ЭТИМ ФОРМАЛЬНЫМ ОПИСАНИЕМ СЕМАНТИКИ ТОЧКИ С ЗАПЯТОЙ ОХВАТЫВАЕТСЯ 
НАШЕ ИНТУИТИВНОЕ ПРЕДСТАВЛЕНИЕ О НЕЙ (ЕСЛИ ОНО У НАС ЕСТЬ!), 
Т. Е. ЧТО СОСТАВНАЯ КОНСТРУКЦИЯ "$S1;S2$"  МОЖЕТ БЫТЬ РЕАЛИЗОВАНА 
НО ПРАВИЛУ "СНАЧАЛА ЗАПУСТИТЬ КОНСТРУКЦИЮ $S1$ И Пo ОКОНЧАНИИ ЕЕ 
РАБОТЫ ЗАПУСТИТЬ $S2$".  в САМОМ ДЕЛЕ, В НАШЕМ ОПРЕДЕЛЕНИИ 
$\wp ("S1; S2", R)$ МЫ  ПРЕДСТАВЛЯЕМ $R$-ПОСТУСЛОВИЕ ДЛЯ 
СОСТАВНОЙ КОНСТРУКЦИИ --- КАК ПОСТУСЛОВИЕ К ПРЕОБРАЗОВАТЕЛЮ 
ПРЕДИКАТОВ ДЛЯ $S2$, И ЭТИМ ОТРАЖАЕТСЯ ТО, ЧТО ОБЩАЯ РАБОТА 
КОНСТРУКЦИИ "$S1; S2$" МОЖЕТ ЗАКОНЧИТЬСЯ С ОКОНЧАНИЕМ РАБОТЫ 
$S2$. сООТВЕТСТВУЮЩЕЕ СЛАБЕЙШЕЕ ПРЕДУСЛОВИЕ ДЛЯ $S2$, Т. Е. 
$\wp(S2, R)$, ПРЕДСТАВЛЯЕТСЯ КАК ПОСТУСЛОВИЕ К ПРЕОБРАЗОВАТЕЛЮ 
ПРЕДИКАТОВ ДЛЯ $S1$; ТЕМ САМЫМ МЫ ЯВНО ОТОЖДЕСТВЛЯЕМ НАЧАЛЬНОЕ 
СОСТОЯНИЕ ДЛЯ $S2$ С КОНЕЧНЫМ СОСТОЯНИЕМ ДЛЯ $S1$. оДНАКО ИМЕННО 
ТАК И БЫВАЕТ, ЕСЛИ РАБОТА $S1$ НЕПОСРЕДСТВЕННО ПРЕДШЕСТВУЕТ ВО 
ВРЕМЕНИ ЗАПУСКУ $S2$.

чТОБЫ УДОСТОВЕРИТЬСЯ В ЭТОМ, РАССМОТРИМ ПРИМЕР. пУСТЬ "$S1; S2$" 
ПРЕДСТАВЛЯЕТ СОБОЙ
$$ 
"a:=a+b;  b:=a*b" 
$$
И ПУСТЬ НАШИМ ПОСТУСЛОВИЕМ ЯВЛЯЕТСЯ НЕКОТОРЫЙ ПРЕДИКАТ $R(a, b)$; 
В ТАКОМ СЛУЧАЕ
$$ 
\eqalign{
\wp (S2, R (a, b) ) &=\wp ("b:= a * b", R (a, b))\cr
&=R(a. c*b) \cr
}
$$ 
И
$$ 
\eqalign{
\wp("S1; S2", R(a, b))&=\wp(S1,\wp(S2, R(a, b)))\cr
&=\wp(S1,R(a, a*b))\cr
&=\wp( "a:= a + b ", R ( a, a * b))\cr
&=R(a+b, (a+b)*b)\cr
}
$$
Т.Е МЫ МОЖЕМ ГАРАНТИРОВАТЬ ВЫПОЛНЕНИЕ ОТНОШЕНИЯ $R$ МЕЖДУ 
КОНЕЧНЫМИ ЗНАЧЕНИЯМИ $a$ И $b$ ПРИ УСЛОВИИ, ЧТО ПЕРВОНАЧАЛЬНО ТО 
ЖЕ ОТНОШЕНИЕ ВЫПОЛНЯЕТСЯ МЕЖДУ $a+b$ И $(a+b)*b$ СООТВЕТСТВЕННО.

и НАКОНЕЦ, ПОСКОЛЬКУ ФУНКЦИОНАЛЬНАЯ КОМПОЗИЦИЯ ОБЛАДАЕТ 
СВОЙСТВОМ АССОЦИАТИВНОСТИ, ТО НЕ ИМЕЕТ ЗНАЧЕНИЯ, БУДЕМ ЛИ МЫ 
РАЗБИРАТЬ "S1; S2; S3" КАК "[S1; S2]; S3" ИЛИ ЖЕ КАК 
"S1;[S2;S3]". иНАЧЕ ГОВОРЯ, МЫ ИМЕЕМ ПОЛНОЕ ПРАВО ТРАКТОВАТЬ 
ТОЧКУ С ЗАПЯТОЙ КАК СИМВОЛ СОЧЛЕНЕНИЯ; ПОЭТОМУ НЕ ВОЗНИКАЕТ 
НИКАКОЙ НЕОПРЕДЕЛЕННОСТИ, КОГДА МЫ ВЫПИСЫВАЕМ ОПЕРАТОРНЫЙ 
СПИСОК ВИДА "$S_1$; $S_2$; $S_3$; \dots ; $S_n$", И МЫ БУДЕМ 
БЕЗ СТЕСНЕНИЯ ПОСТУПАТЬ ТАК, КОГДА ДЛЯ ЭТОГО ПРЕДСТАВЯТСЯ 
ПОДХОДЯЩИЕ СЛУЧАИ. 

{\bf уПРАЖНЕНИЕ.}
пРОВЕРЬТЕ, ЧТО КОНСТРУКЦИИ 
$$ 
"x1:=E1; x2:=E2"\hbox{ И }"x2:=E2; x1:=E1" 
$$
СЕМАНТИЧЕСКИ ЭКВИВАЛЕНТНЫ, ЕСЛИ ПЕРЕМЕННАЯ $x1$ НЕ 
ВСТРЕЧАЕТСЯ В ВЫРАЖЕНИИ $E2$, А ПЕРЕМЕННАЯ $x2$ НЕ 
ВСТРЕЧАЕТСЯ В ВЫРАЖЕНИИ $E1$. нА САМОМ ДЕЛЕ, В ТАКОМ СЛУЧАЕ 
ОБЕ ЭТИ КОНСТРУКЦИИ СЕМАНТИЧЕСКИ ЭКВИВАЛЕНТНЫ ОДНОВРЕМЕННОМУ 
ПРИСВАИВАНИЮ "$x1, x2:=E1, E2$". (эТА ЭКВИВАЛЕНТНОСТЬ 
ЯВЛЯЕТСЯ ОДНИМ ИЗ АРГУМЕНТОВ В ПОЛЬЗУ ОДНОВРЕМЕННОГО 
ПРИСВАИВАНИЯ; ЕЕ ПРИМЕНЕНИЕ ПОЗВОЛЯЕТ НАМ ИЗБЕЖАТЬ 
ИЗБЫТОЧНОСТИ ПОСЛЕДОВАТЕЛЬНОЙ ЗАПИСИ, БОЛee ТОГО, ПРИ 
ОДНОВРЕМЕННОМ ПРИСВАИВАНИИ СТАНОВИТСЯ ОЧЕВИДНЫМ, ЧТО ДВА 
ВЫРАЖЕНИЯ $E1$ И $E2$ МОГУТ ВЫЧИСЛЯТЬСЯ ОДНОВРЕМЕННО, И ЭТО 
ПОСЛЕДНЕЕ ОБСТОЯТЕЛЬСТВО ПРЕДСТАВЛЯЕТ ИНТЕРЕС ПРИ НЕКОТОРЫХ 
МЕТОДИКАХ РЕАЛИЗАЦИИ. пОМИМО ТОГО, БЫТЬ МОЖЕТ, БОЛЕЕ 
ИНТЕРЕСНАЯ ВОЗМОЖНОСТЬ, ЧТО КОНСТРУКЦИЯ "$x1, x2:=E1, E2$" НЕ 
ОКАЖЕТСЯ СЕМАНТИЧЕСКИ ЭКВИВАЛЕНТНОЙ НИ КОНСТРУКЦИИ 
"$x1:=E1; x2:=E2$", НИ КОНСТРУКЦИИ "$x2:=E2; x1:=E1$".) (кОНЕЦ 
УПРАЖНЕНИЯ.)

дО ВВЕДЕНИЯ ТОЧКИ С ЗАПЯТОЙ МЫ МОГЛИ ПИСАТЬ ТОЛЬКО 
ОДНООПЕРАТОРНЫЕ ПРОГРАММЫ; С ПОМОЩЬЮ ТОЧКИ С ЗАПЯТОЙ МЫ 
ОБРЕЛИ СПОСОБНОСТЬ ПИСАТЬ ПРОГРАММЫ В ВИДЕ СОЧЛЕНЕНИЯ ИЗ $n$, 
$(n>0)$ ОПЕРАТОРОВ: "$S_1$; $S_2$; $S_3$; \dots; 
$S_n$". еСЛИ ИСКЛЮЧИТЬ ПРОМЕЖУТОЧНУЮ НЕЗАВЕРШЕННОСТЬ, 
ВЫПОЛНЕНИЕ КАЖДОЙ ПРОГРАММЫ ВСЕГДА ОЗНАЧАЕТ ВРЕМЕННУЮ 
ПОСЛЕДОВАТЕЛЬНОСТЬ ВЫПОЛНЕНИЙ $n$ ОПЕРАТОРОВ, СНАЧАЛА 
$S_1$, ПОТОМ $S_2$ И ТАК ДАЛЕЕ ДО $S_n$ ВКЛЮЧИТЕЛЬНО. 
оДНАКО ИЗ НАШЕГО ПРИМЕРА ИГРЫ НА ЛИСТЕ КАРТОНА, РЕАЛИЗУЮЩЕЙ 
АЛГОРИТМ еВКЛИДА, МЫ ЗНАЕМ, ЧТО ДОЛЖНЫ УМЕТЬ ОПИСЫВАТЬ БОЛЕЕ 
ШИРОКИЙ КЛАСС "ПРАВИЛ ИГРЫ": ВСЯКАЯ ИГРА БУДЕТ СОСТОЯТЬ ИЗ 
ПОСЛЕДОВАТЕЛЬНОСТИ ПЕРЕМЕЩЕНИЙ, ГДЕ КАЖДОЕ ПЕРЕМЕЩЕНИЕ ИМЕЕТ 
ВИД ЛИБО "$x:=x-y$", ЛИБО "$y:=y-x$", НО СПОСОБ ЧЕРЕДОВАНИЯ 
ЭТИХ ПЕРЕМЕЩЕНИЙ ВО ВРЕМЕНИ И ДАЖЕ ИХ ОБЩЕЕ ЧИСЛО БУДУТ 
ИЗМЕНЯТЬСЯ ОТ ИГРЫ К ИГРЕ; ОН ЗАВИСИТ ОТ НАЧАЛЬНОГО ПОЛОЖЕНИЯ 
ФИШКИ, ОН ЗАВИСИТ ОТ НАЧАЛЬНОГО СОСТОЯНИЯ СИСТЕМЫ. еСЛИ ТОЧКА 
С ЗАПЯТОЙ ЯВЛЯЕТСЯ ЕДИНСТВЕННЫМ НАШИМ СРЕДСТВОМ ДЛЯ 
СОСТАВЛЕНИЯ НОВОГО ЦЕЛОГО ИЗ ЗАДАННЫХ ЧАСТЕЙ, ТО МЫ НЕ В 
СОСТОЯНИИ ЭТОГО ВЫРАЗИТЬ, И ПОЭТОМУ НАМ НЕОБХОДИМО ИСКАТЬ 
НЕЧТО НОВОЕ.

пОКА ТОЧКА С ЗАПЯТОЙ ОСТАЕТСЯ ЕДИНСТВЕННЫМ ИМЕЮЩИМСЯ В НАШЕМ 
РАСПОРЯЖЕНИИ СРЕДСТВОМ СОЕДИНЕНИЯ, ЕДИНСТВЕННЫМ 
ОБСТОЯТЕЛЬСТВОМ, ПРИ КОТОРОМ ПРОИСХОДИТ ЗАПУСК ОДНОЙ ИЗ 
СОСТАВЛЯЮЩИХ КОНСТРУКЦИЙ $S_i\; (i>1)$, ЯВЛЯЮТСЯ ПРАВИЛЬНОЕ 
ЗАВЕРШЕНИЕ РАБОТЫ (ЛЕКСИКОГРАФИЧЕСКИ) ПРЕДШЕСТВУЮЩЕЙ 
КОНСТРУКЦИИ. чТОБЫ ДОБИТЬСЯ НУЖНОЙ НАМ ГИБКОСТИ, НЕОБХОДИМО 
ОБЕСПЕЧИТЬ ВОЗМОЖНОСТЬ ЗАПУСКА ТОЙ ИЛИ ИНОЙ (ПОД) КОНСТРУКЦИИ 
В ЗАВИСИМОСТИ ОТ ТЕКУЩЕГО СОСТОЯНИЯ СИСТЕМЫ. дЛЯ ЭТОГО МЫ 
ВВЕДЕМ --- В ДВА ЭТАПА --- ПОНЯТИЕ "ОХРАНЯЕМОЙ КОМАНДЫ", 
СИНТАКСИС КОТОРОЙ ЗАДАЕТСЯ СЛЕДУЮЩИМ ОБРАЗОМ:
$$
\eqalign{
\<ОХРАНЯЮЩИЙ ЗАГОЛОВОК>&::=\<ЛОГИЧЕСКОЕ ВЫРАЖЕНИЕ>\to\cr
&\<ОПЕРАТОР>\cr
\<ОХРАНЯЕМАЯ КОМАНДА>&::=\<ОХРАНЯЮЩИЙ ЗАГОЛОВОК> \{;\cr
&\<ОПЕРАТОР>\}\cr
}
$$
ГДЕ ФИГУРНЫЕ СКОБКИ "$\{$" И "$\}$" СЛЕДУЕТ ЧИТАТЬ ТАК: 
"conpОВОЖДАЕТСЯ ЛЮБЫМ ЧИСЛОМ (БЫТЬ МОЖЕТ, НУЛЕМ) ЭКЗЕМПЛЯРОВ 
СОДЕРЖИМОГО СКОБОК".

(дРУГОЙ ВАРИАНТ СИНТАКСИСА ОХРАНЯЕМОЙ КОМАНДЫ МОЖЕТ ВЫГЛЯДЕТЬ 
ТАК:
$$
\eqalign{
\<СПИСОК ОПЕРАТОРОВ>&::=\<ОПЕРАТОР> \{;\<ОПЕРАТОР>\}\cr 
\<ОХРАНЯЕМАЯ КОМАНДА>&::=\<ЛОГИЧЕСКОЕ ВЫРАЖЕНИЕ>\to\cr
&\<СПИСОК ОПЕРАТОРОВ>\cr
}
$$
оДНАКО ПО ПРИЧИНАМ, КОТОРЫЕ НЕ ДОЛЖНЫ НАС ТЕПЕРЬ 
ИНТЕРЕСОВАТЬ, Я ПРЕДПОЧИТАЮ СИНТАКСИС, В КОТОРОМ ВВОДИТСЯ 
ПОНЯТИЕ ОХРАНЯЮЩЕГО ЗАГОЛОВКА).

в ЭТОМ СОЧЕТАНИИ ЛОГИЧЕСКОЕ ВЫРАЖЕНИЕ, ПРЕДШЕСТВУЮЩЕЕ 
СТРЕЛКЕ, НАЗЫВАЕТСЯ "ПРЕДОХРАНИТЕЛЕМ". иДЕЯ СОСТОИТ В ТОМ, 
ЧТО СЛЕДУЮЩИЙ ЗА СТРЕЛКОЙ СПИСОК ОПЕРАТОРОВ БУДЕТ ВЫПОЛНЯТЬСЯ 
ЛИШЬ ПРИ ТОМ УСЛОВИИ, ЧТО ОБЕСПЕЧЕНА НАЧАЛЬНАЯ ИСТИННОСТЬ 
ЗНАЧЕНИЯ СООТВЕТСТВУЮЩЕГО ПРЕДОХРАНИТЕЛЯ.  пРЕДОХРАНИТЕЛЬ 
ПОЗВОЛЯЕТ НАМ ИЗБЕЖАТЬ ВЫПОЛНЕНИЯ СПИСКА  ОПЕРАТОРОВ ПРИ ТЕХ 
ПЕРВОНАЧАЛЬНЫХ ОБСТОЯТЕЛЬСТВАХ, КОГДА ЭТО ПОЛНЕНИЕ ОКАЗАЛОСЬ 
БЫ НЕЖЕЛАТЕЛЬНЫМ ИЛИ (ЕСЛИ УПОТРЕБЛЯЮТСЯ ЧАСТИЧНО 
ОПРЕДЕЛЕННЫЕ ОПЕРАЦИИ) НЕВОЗМОЖНЫМ.

иСТИННОСТЬ ЗНАЧЕНИЯ ПРЕДОХРАНИТЕЛЯ ЯВЛЯЕТСЯ НЕОБХОДИМЫМ 
ПРЕДВАРИТЕЛЬНЫМ УСЛОВИЕМ ДЛЯ ВЫПОЛНЕНИЯ ОХРАНЯЕМОЙ КОМАНДЫ 
КАК ЦЕЛОГО; ЭТО УСЛОВИЕ, РАЗУМЕЕТСЯ, НЕ ЯВЛЯЕТСЯ ДОСТАТОЧНЫМ, 
ПОСКОЛЬКУ ТЕМ ИЛИ ИНЫМ СПОСОБОМ --- С ДВУМЯ ТАКИМИ СПОСОБАМИ 
МЫ ВСТРЕТИМСЯ --- ДО НЕГО ЕЩЕ ДОЛЖНА ДОЙТИ ПОТЕНЦИАЛЬНАЯ 
"ОЧЕРЕДЬ". иМЕННО ПОЭТОМУ ОХРАНЯЕМАЯ КОМАНДА НЕ 
РАССМАТРИВАЕТСЯ КАК ОПЕРАТОР: ОПЕРАТОР БЕЗОГОВОРОЧНО
ВЫПОЛНЯЕТСЯ, КОГДА НАСТУПАЕТ ЕГО ОЧЕРЕДЬ, А ОХРАНЯЕМАЯ 
КОМАНДА МОЖЕТ ИСПОЛЬЗОВАТЬСЯ В КАЧЕСТВЕ СТРОИТЕЛЬНОГО БЛОКА 
ДЛЯ ОПЕРАТОРА. еСЛИ ГОВОРИТЬ ТОЧНЕЕ, МЫ ПРЕДЛОЖИМ ДВА 
РАЗЛИЧНЫХ СПОСОБА СОСТАВЛЕНИЯ ОПЕРАТОРА ИЗ НАБОРА ОХРАНЯЕМЫХ 
КОМАНД.

 пОСЛЕ НЕКОТОРОГО ОСМЫСЛИВАНИЯ РАССМОТРЕНИЕ НАБОРА ОХРАНЯЕМЫХ 
КОМАНД ПРЕДСТАВЛЯЕТСЯ ВПОЛНЕ ЕСТЕСТВЕННЫМ. пРЕДПОЛОЖИМ, ЧТО 
НАМ ТРЕБУЕТСЯ ПОСТРОИТЬ КОНСТРУКЦИЮ, ТАКУЮ, ЧТОБЫ ПРИ 
УСЛОВИИ, ЧТО НАЧАЛЬНОЕ СОСТОЯНИЕ УДОВЛЕТВОРЯЕТ $Q$, КОНЕЧНОЕ 
СОСТОЯНИЕ УДОВЛЕТВОРЯЛО $R$. пРЕДПОЛОЖИМ ДАЛЕЕ, ЧТО НАМ НЕ 
УДАЕТСЯ НАЙТИ ЕДИНЫЙ СПИСОК ОПЕРАТОРОВ, КОТОРЫЙ ВЫПОЛНЯЛ БЫ 
ЭТУ РАБОТУ В ЛЮБЫХ СЛУЧАЯХ. (еСЛИ БЫ ТАКОЙ СПИСОК  ОПЕРАТОРОВ 
СУЩЕСТВОВАЛ, ТО МЫ ИМЕННО ЕГО БЫ И ИСПОЛЬЗОВАЛИ И НЕ ВОЗНИКЛО 
БЫ ПОТРЕБНОСТИ В ОХРАНЯЕМЫХ КОМАНДАХ.) оДНАКО НАМ МОЖЕТ 
УДАСТЬСЯ НАЙТИ НЕСКОЛЬКО СПИСКОВ ОПЕРАТОРОВ, КАЖДЫЙ ИЗ 
КОТОРЫХ БУДЕТ ВЫПОЛНЯТЬ НУЖНУЮ РАБОТУ ДЛЯ НЕКoero 
ПОДМНОЖЕСТВА ВОЗМОЖНЫХ НАЧАЛЬНЫХ СОСТОЯНИЙ. к КАЖДОМУ ИЗ ЭТИХ 
СПИСКОВ ОПЕРАТОРОВ МЫ МОЖЕМ ПРИСОЕДИНИТЬ В КАЧЕСТВЕ 
ПРЕДОХРАНИТЕЛЯ ЛОГИЧЕСКОЕ ВЫРАЖЕНИЕ, ХАРАКТЕРИЗУЮЩЕЕ ТО 
ПОДМНОЖЕСТВО, ДЛЯ КОТОРОГО ЭТОТ СПИСОК ПОДХОДИТ, И ЕСЛИ У НАС 
ИМЕЕТСЯ ВПОЛНЕ ДОСТАТОЧНЫЙ НАБОР ПРЕДОХРАНИТЕЛЕЙ, ТАКОЙ, ЧТО 
ИЗ ИСТИННОСТИ $Q$ ВСЕГДА ЛОГИЧЕСКИ СЛЕДУЕТ ИСТИННОСТЬ 
ЗНАЧЕНИЯ ХОТЯ БЫ ОДНОГО ПРЕДОХРАНИТЕЛЯ, ТО ДЛЯ КАЖДОГО 
НАЧАЛЬНОГО СОСТОЯНИЯ, УДОВЛЕТВОРЯЮЩЕГО $Q$, МЫ РАСПОЛАГАЕМ 
КОНСТРУКЦИЕЙ, КОТОРАЯ ПЕРЕВЕДЕТ СИСТЕМУ В СОСТОЯНИЕ, 
УДОВЛЕТВОРЯЮЩЕЕ УСЛОВИЮ $R$, ПРИЧЕМ ЭТОЙ КОНСТРУКЦИЕЙ СЛУЖИТ 
ОДНА ИЗ ОХРАНЯЕМЫХ КОМАНД, ЧЕЙ ПРЕДОХРАНИТЕЛЬ ИМЕЛ НАЧАЛЬНОЕ 
ЗНАЧЕНИЕ "ИСТИНА". 
чТОБЫ ВЫРАЗИТЬ ЭТО, ОПРЕДЕЛИМ СНАЧАЛА
$$ 
\<НАБОР ОХРАНЯЕМЫХ КОМАНД>::=\<ОХРАНЯЕМАЯ КОМАНДА> \{\wbox 
\<ОХРАНЯЕМАЯ КОМАНДА>\} 
$$ 
%\ wbox --- white box vs black border: []
ГДЕ СИМВОЛ "\wbox" ВЫСТУПАЕТ В РОЛИ РАЗДЕЛИТЕЛЯ ВАРИАНТОВ, В 
ОСТАЛЬНОМ НЕ УПОРЯДОЧЕННЫХ. оДИН ИЗ СПОСОБОВ ФОРМИРОВАНИЯ 
ОПЕРАТОРА ИЗ НАБОРА ОХРАНЯЕМЫХ КОМАНД СОСТОИТ В ТОМ, ЧТОБЫ 
ВКЛЮЧИТЬ ТАКОЙ НАБОР В ПАРУ СКОБОК "\kwd{if} \dots \kwd{fi}", ПРИ ЭТОМ НАШ 
СИНТАКСИС ДЛЯ СИНТАКСИЧЕСКОЙ КАТЕГОРИИ, ИМЕНУЕМОЙ "ОПЕРАТОР", 
ПОПОЛНЯЕТСЯ СЛЕДУЮЩЕЙ ФОРМОЙ:
$$ 
 \<ОПЕРАТОР>::=\kwd{if}\<НАБОР ОХРАНЯЕМЫХ КОМАНД> \kwd{fi} 
$$
тАКИМ ОБРАЗОМ, МЫ ПОЛУЧАЕМ ВОЗМОЖНОСТЬ ОБRЕДИНИТЬ НЕСКОЛЬКО 
ОХРАНЯЕМЫХ КОМАНД В НОВУЮ КОНСТРУКЦИЮ. мЫ МОЖЕМ ПРЕДСТАВИТЬ 
СЕБЕ РАБОТУ, КОТОРАЯ ПРОИЗОЙДЕТ ПОСЛЕ ЗАПУСКА ЭТОЙ 
КОНСТРУКЦИИ, СЛЕДУЮЩИМ ОБРАЗОМ. вЫБИРАЕТСЯ ОДНА ИЗ
ОХРАНЯЕМЫХ КОМАНД, ЧЕЙ ПРЕДОХРАНИТЕЛЬ ЯВЛЯЕТСЯ ИСТИННЫМ, И 
ЗАПУСКАЕТСЯ ЕЕ СПИСОК ОПЕРАТОРОВ.

пРЕЖДЕ ЧЕМ МЫ ПЕРЕЙДЕМ К ИЗЛОЖЕНИЮ ФОРМАЛЬНОГО ОПИСАНИЯ 
СЕМАНТИКИ НАШЕЙ НОВОЙ КОНСТРУКЦИИ, УМЕСТНО СДЕЛАТЬ ТРИ 
ЗАМЕЧАНИЯ.                                    

\medskip
\item{1.} пРЕДПОЛАГАЕТСЯ, ЧТО ВСЕ ПРЕДОХРАНИТЕЛИ ОПРЕДЕЛЕНЫ; 
ЕСЛИ ЭТО НЕ ТАК, Т. Е. ВЫЧИСЛЕНИЕ КАКОГО-ТО ПРЕДОХРАНИТЕЛЯ 
МОЖЕТ ПРИИЕСТИ К РАБОТЕ БЕЗ ПРАВИЛЬНОГО ЗАВЕРШЕНИЯ, ТО 
ДОПУСКАЕТСЯ, ЧТО И ВСЯ КОНСТРУКЦИЯ НЕ СМОЖЕТ ПРАВИЛЬНО 
ЗАВЕРШИТЬ СВОЮ РАБОТУ.

\item{2.} вООБЩЕ ГОВОРЯ, НАША КОНСТРУКЦИЯ ПРИВЕДЕТ К 
НЕДЕТЕРМИНИРОВАННОСТИ ПРИ ТЕХ НАЧАЛЬНЫХ СОСТОЯНИЯХ, ДЛЯ 
КОТОРЫХ ИСТИННЫ ЗНАЧЕНИЯ БОЛЕЕ ЧЕМ ОДНОГО ПРЕДОХРАНИТЕЛЯ, 
ПОСКОЛЬКУ ОСТАЕТСЯ НЕОПРЕДЕЛЕННЫМ, КАКОЙ ИЗ СООТВЕТСТВУЮЩИХ 
СПИСКОВ ОПЕРАТОРОВ БУДЕТ ТОГДА ВЫБИРАТЬСЯ ДЛЯ ЗАПУСКА. 
HИКАКОЙ НЕДЕТЕРМИНИРОВАННОСТИ НЕ ВОЗНИКАЕТ, ЕСЛИ ВСЕ 
ПРЕДОХРАНИТЕЛИ ПОПАРНО ИСКЛЮЧАЮТ ДРУГ ДРУГА.

\item{3.} еСЛИ НАЧАЛЬНОЕ СОСТОЯНИЕ ТАКОВО, ЧТО НИ ОДИН ИЗ  
ПРЕДОХРАНИТЕЛЕЙ НЕ ЯВЛЯЕТСЯ ИСТИНОЙ, ТО МЫ ВСТРЕЧАЕМСЯ С 
НАЧАЛЬНЫМ СОСТОЯНИЕМ, ДЛЯ КОТОРОГО НЕ ПОДХОДИТ НИ ОДИН ИЗ  
ВАРИАНТОВ, А СЛЕДОВАТЕЛЬНО, И ВСЯ КОНСТРУКЦИЯ В ЦЕЛОМ. зАПУСК 
ПРИ ТАКОМ НАЧАЛЬНОМ СОСТОЯНИИ ПРИВЕДЕТ К ОТКАЗУ.
\medskip

{\sl зАМЕЧАНИЕ.} еСЛИ МЫ ДОПУСКАЕМ ТАКЖЕ И ПУСТОЙ НАБОР 
ОХРАНЯЕМЫХ КОМАНД, ТО ОПЕРАТОР "\kwd{if}-\kwd{fi}" СЕМАНТИЧЕСКИ 
ЭКВИВАЛЕНТЕН НАШЕМУ ПРЕЖНЕМУ ОПЕРАТОРУ "ОТКАЗАТЬ". {\sl(кОНЕЦ 
ЗАМЕЧАНИЯ.)}

(в СЛЕДУЮЩЕМ ФОРМАЛЬНОМ ОПРЕДЕЛЕНИИ СЛАБЕЙШЕГО ПРЕДУСЛОВИЯ 
ДЛЯ КОНСТРУКЦИИ \kwd{if}-\kwd{fi} МЫ ОГРАНИЧИМСЯ СЛУЧАЕМ, КОГДА ВСЕ 
ПРЕДОХРАНИТЕЛИ ЯВЛЯЮТСЯ ПОЛНОСТЬЮ ОПРЕДЕЛЕННЫМИ ФУНКЦИЯМИ. 
еСЛИ ЭТО НЕ ТАК, ТО НУЖНО ПЕРЕПИСАТЬ ВЫРАЖЕНИЕ С 
ИСПОЛЬЗОВАНИЕМ СИМВОЛА $\cand$ ПРИ ДОПОЛНИТЕЛЬНОМ ТРЕБОВАНИИ, 
ЧТОБЫ НАЧАЛЬНОЕ СОСТОЯНИЕ ПРИНАДЛЕЖАЛО ОБЛАСТИ  ОПРЕДЕЛЕНИЯ 
ВСЕХ ПРЕДОХРАНИТЕЛЕЙ.) 
оБОЗНАЧИМ ЧЕРЕЗ "\kwd{IF}" ОПЕРАТОР  
\footnote{1} {SL- АББРЕВИАТУРА ДЛЯ Statement List --- СПИСОК 
ОПЕРАТОРОВ. ---{\sl пРИМ. ПЕРЕВ.} }
$$ 
 \kwd{if} B_1 \to SL_1 \wbox B_2\to SL_2 \wbox \dots \wbox 
B_n \to SL_n 
 \kwd{fi} 
$$
тОГДА ДЛЯ ЛЮБОГО ПОСТУСЛОВИЯ $R$ 
$$ 
\wp(IF, R) = (\exists j : 1\le j \le n: B_j) \and (\forall 
j : 1\le j \le n : B_j \Rightarrow wp(SL_j,R)) 
$$ % A И E --- ЭТО \forall И \exist?

эТУ ФОРМУЛУ СЛЕДУЕТ ЧИТАТЬ ТАК: $\wp(IF, R)$ ЯВЛЯЕТСЯ ИСТИНОЙ 
ДЛЯ КАЖДОЙ ТАКОЙ ТОЧКИ В ПРОСТРАНСТВЕ СОСТОЯНИЙ, ГДЕ
ХОТЯ БЫ ОДНО ЗНАЧЕНИЕ $j$ ИЗ ОТРЕЗКА $1\le j \le n$, ТАКОЕ, 
ЧТО $B_j$ ЯВЛЯЕТСЯ ИСТИНОЙ, И ГДЕ, КРОМЕ ТОГО, ДЛЯ ВСЕХ $j$ 
ИЗ ОТРЕЗКА $1\le j \le n$, ТАКИХ, ЧТО $B_j$ --- ИСТИНА, 
ЗНАЧЕНИЕ $wp(SL_j, R)$ ТАКЖЕ ЯВЛЯЕТСЯ ИСТИНОЙ. иСПОЛЬЗУЯ 
СИМВОЛ "\dots", КАК МЫ УЖЕ ДЕЛАЛИ ПРИ ОПИСАНИИ САМОГО 
ОПЕРАТОРА IF, МЫ МОЖЕМ ПРЕДЛОЖИТЬ ИНУЮ ФОРМУ ОПРЕДЕЛЕНИЯ:
$$ 
\eqalign{
\wp (IF, R) =&(B_1 \or B_2 \or \ldots \or B_n) \and \cr                   
&(B_1 \Rightarrow \wp(SL_1, R)) \and \cr                   
&(B_2 \Rightarrow \wp(SL_2, R)) \and \ldots \and \cr                   
&(B_n \Rightarrow \wp(SL_n, R) )\cr 
}
$$

пОНИМАНИЕ ЭТИХ ФОРМУЛ НЕ ПРЕДСТАВЛЯЕТ ОСОБОГО ТРУДА. 
тРЕБОВАНИЕМ, ЧТОБЫ ЗНАЧЕНИЕ ХОТЯ БЫ ОДНОГО ПРЕДОХРАНИТЕЛЯ 
ЯВЛЯЛОСЬ ИСТИНОЙ, ОТРАЖАЕТСЯ ФАКТ ОТКАЗА В СЛУЧАЕ, КОГДА 
ЗНАЧЕНИЯ ДЛЯ ВСЕХ ПРЕДОХРАНИТЕЛЕЙ ЛОЖНЫ. кРОМЕ ТОГО, МЫ 
ТРЕБУЕМ ДЛЯ КАЖДОГО НАЧАЛЬНОГО СОСТОЯНИЯ, УДОВЛЕТВОРЯЮЩЕГО 
$\wp(IF,R)$, ЧТОБЫ ВЫПОЛНЯЛОСЬ $B_j \Rightarrow wp(SL_j, R)$ ДЛЯ 
ВСЕХ $j$. дЛЯ ТЕХ ЗНАЧЕНИЙ $j$, ПРИ КОТОРЫХ $в_j$  --- 
ЛОЖЬ, ЭТО СЛЕДОВАНИЕ ИСТИННО НЕЗАВИСИМО ОТ ЗНАЧЕНИЯ 
$wp(SL_j, R)$, Т. Е. ДЛЯ ТАКИХ ЗНАЧЕНИЙ $j$, РАЗУМЕЕТСЯ, 
БЕЗРАЗЛИЧНО, ЧТО БУДЕТ ДЕЛАТЬ $SL_j$. пРИ НАШЕЙ РЕАЛИЗАЦИИ 
ЭТО УЧИТЫВАЕТСЯ В ТОМ, ЧТО ДЛЯ ЗАПУСКА НЕ БЕРЕТСЯ $SL_j$ С 
ЛОЖНЫМ НАЧАЛЬНЫМ ЗНАЧЕНИЕМ ПРЕДОХРАНИТЕЛЯ $B_j$. пРИ ТЕХ 
ЗНАЧЕНИЯХ $j$, ДЛЯ КОТОРЫХ $B_j$ --- ИСТИНА, ДАННОЕ 
СЛЕДОВАНИЕ МОЖЕТ БЫТЬ ИСТИНОЙ ТОЛЬКО В ТОМ СЛУЧАЕ, КОГДА 
$wp(SL_j,R)$ ТАКЖЕ ЯВЛЯЕТСЯ ИСТИНОЙ. пОСКОЛЬКУ НАШЕ 
ФОРМАЛЬНОЕ ОПРЕДЕЛЕНИЕ ТРЕБУЕТ ИСТИННОСТИ СЛЕДОВАНИЯ ПРИ ВСЕХ 
ЗНАЧЕНИЯХ $j$, ТО РЕАЛИЗАЦИЯ НА САМОМ ДЕЛЕ ИМЕЕТ СВОБОДУ 
ВЫБОРА, КОГДА БОЛЕЕ ЧЕМ ОДИН ПРЕДОХРАНИТЕЛЬ ЯВЛЯЕТСЯ ИСТИНОЙ. 
кОНСТРУКЦИЯ \kwd{if}-\kwd{fi} ПРЕДСТАВЛЯЕТ СОБОЙ ТОЛЬКО ОДИН ИЗ 
ДВУХ ВОЗМОЖНЫХ СПОСОБОВ ПОСТРОЕНИЯ ОПЕРАТОРА ИЗ НАБОРА 
ОХРАНЯЕМЫХ КОМАНД. в КОНСТРУКЦИИ \kwd{if}-\kwd{fi} СОСТОЯНИЕ, ПРИ 
КОТОРОМ ВСЕ ПРЕДОХРАНИТЕЛИ ИМЕЮТ ЛОЖНЫЕ ЗНАЧЕНИЯ, ВЕДЕТ К 
ОТКАЗУ. в НАШЕЙ ВТОРОЙ ФОРМЕ МЫ РАЗРЕШАЕМ, ЧТОБЫ СОСТОЯНИЕ, 
ПРИ КОТОРОМ НЕТ НИ ОДНОГО ПРЕДОХРАНИТЕЛЯ С ИСТИННЫМ 
ЗНАЧЕНИЕМ, ПРИВОДИЛО К ПРАВИЛЬНОМУ ЗАВЕРШЕНИЮ; ПОСКОЛЬКУ В 
ЭТОЙ СИТУАЦИИ НЕ ЗАПУСКАЕТСЯ НИКАКОЙ СПИСОК ОПЕРАТОРОВ, 
ВПОЛНЕ ЕСТЕСТВЕННО, ЧТО ПРИ ЭТОМ ВОЗНИКАЕТ СЕМАНТИЧЕСКАЯ 
ЭКВИВАЛЕНТНОСТЬ С ПУСТЫМ ОПЕРАТОРОМ. оДНАКО ЭТО РАЗРЕШЕНИЕ НА 
ПРАВИЛЬНОЕ ЗАВЕРШЕНИЕ, КОГДА НЕТ НИ ОДНОГО ПРЕДОХРАНИТЕЛЯ С 
ИСТИННЫМ ЗНАЧЕНИЕМ, ДОПОЛНЯЕТСЯ ТЕМ, ЧТО РАБОТЕ НЕ 
РАЗРЕШАЕТСЯ ЗАВЕРШАТЬСЯ, ПОКА ХОТЯ БЫ ОДИН ПРЕДОХРАНИТЕЛЬ 
ЯВЛЯЕТСЯ ИСТИНОЙ. иТАК, ПОСЛЕ ЗАПУСКА ПРОВЕРЯЮТСЯ ВСЕ 
ПРЕДОХРАНИТЕЛИ. еСЛИ  НЕТ НИ ОДНОГО ИСТИННОГО ЗНАЧЕНИЯ 
ПРЕДОХРАНИТЕЛЯ, ТО РАБОТА ЗАВЕРШАЕТСЯ; ЕСЛИ ЖЕ ИМЕЮТСЯ 
ПРЕДОХРАНИТЕЛИ С ИСТИННЫМИ ЗНАЧЕНИЯМИ, ТО ОДИН ИЗ 
СООТВЕТСТВУЮЩИХ СПИСКОВ ОПЕРАТОРОВ ЗАПУСКАЕТСЯ, А ПОСЛЕ ЕГО 
ЗАВЕРШЕНИЯ РЕАЛИЗАЦИЯ СНОВА НАЧИНАЕТ ОБЩУЮ ПРОВЕРКУ ВСЕХ 
ПРЕДОХРАНИТЕЛЕЙ. эТА ВТОРАЯ КОНСТРУКЦИЯ ОБОЗНАЧАЕТСЯ ПУТЕМ 
ЗАКЛЮЧЕНИЯ СПИСКА ОХРАНЯЕМЫХ КОМАНД В ПАРУ СКОБОК "\kwd{do} .. \kwd{od}".

фОРМАЛЬНОЕ ОПРЕДЕЛЕНИЕ СЛАБЕЙШЕГО ПРЕДУСЛОВИЯ ДЛЯ  
КОНСТРУКЦИИ \kwd{do}-\kwd{od} ЯВЛЯЕТСЯ БОЛЕЕ СЛОЖНЫМ, ЧЕМ ДЛЯ 
КОНСТРУКЦИИ \kwd{if}-\kwd{fi}; ДЕЛО В ТОМ, ЧТО ПЕРВОЕ ВЫРАЖАЕТСЯ ЧЕРЕЗ 
ВТОРОЕ. мЫ СНАЧАЛА ПРИВЕДЕМ ЭТО ФОРМАЛЬНОЕ ОПРЕДЕЛЕНИЕ, А 
ЗАТЕМ ОБRЯСНИМ ЕГО. оБОЗНАЧИМ ЧЕРЕЗ "DO" ОПЕРАТОР
$$ 
 \kwd{do} B_1\to SL_1\wbox  B_2\to SL_2 \wbox \ldots \wbox 
B_n \to SL_n \kwd{od} 
$$
И ЧЕРЕЗ "IF" ОПЕРАТОР, ПОЛУЧАЕМЫЙ ПУТЕМ ЗАКЛЮЧЕНИЯ ТОГО ЖЕ 
НАБОРА ОХРАНЯЕМЫХ КОМАНД В ПАРУ СКОБОК "\kwd{if} ... \kwd{fi}". еСЛИ  
УСЛОВИЯ $H_k(R)$ ЗАДАЮТСЯ В ВИДЕ 
$$ 
H_0(R)=R \and \non (\exists j: 1\le j \le n : B_j) 
$$
И ПРИ $k>0$
$$ 
H_k(R)=\wp(IF, H_{k-1} (R)) \or H_0(R) 
$$ 
ТО
$$ 
\wp(DO,R) = (\exists k: k\ge 0 : H_k(R)) 
$$

зДЕСЬ ИНТУИТИВНО МЫ ПОНИМАЕМ $H_k(R)$ СЛЕДУЮЩИМ  ОБРАЗОМ: 
ЭТО СЛАБЕЙШЕЕ ПРЕДУСЛОВИЕ, ТАКОЕ, ЧТО КОНСТРУКЦИЯ \kwd{do}-\kwd{od}
ЗАВЕРШИТ СВОЮ РАБОТУ ПОСЛЕ НЕ БОЛЕЕ ЧЕМ $k$ ВЫБОРОК 
ОХРАНЯЕМЫХ КОМАНД, ПРИЧЕМ СИСТЕМА ОСТАНЕТСЯ В КОНЕЧНОМ 
СОСТОЯНИИ, УДОВЛЕТВОРЯЮЩЕМ ПОСТУСЛОВИЮ $R$.

пРИ $k=0$ ТРЕБУЕТСЯ, ЧТОБЫ КОНСТРУКЦИЯ \kwd{do}-\kwd{od} ЗАВЕРШИЛА 
РАБОТУ, НЕ ПРОИЗВОДЯ ВЫБОРКИ КАКОЙ-ЛИБО ОХРАНЯЕМОЙ КОМАНДЫ, 
Т. Е. НЕ ДОПУСКАЕТСЯ СУЩЕСТВОВАНИЯ КАКОГО-ЛИБО ПРЕДОХРАНИТЕЛЯ 
С ИСТИННЫМ ЗНАЧЕНИЕМ, ЧТО И ВЫРАЖЕНО ВТОРЫМ ЛОГИЧЕСКИМ 
СОМНОЖИТЕЛЕМ. пРИ ЭТОМ НАЧАЛЬНАЯ ИСТИННОСТЬ $R$ ОЧЕВИДНЫМ 
ОБРАЗОМ ЯВЛЯЕТСЯ НЕОБХОДИМЫМ УСЛОВИЕМ ДЛЯ КОНЕЧНОЙ ИСТИННОСТИ 
$R$, ЧТО И ВЫРАЖЕНО ПЕРВЫМ ЛОГИЧЕСКИМ СОМНОЖИТЕЛЕМ.

пРИ $k>0$ НАМ НУЖНО РАЗЛИЧАТЬ ДВА СЛУЧАЯ: ЛИБО НИ ОДИН ИЗ 
ПРЕДОХРАНИТЕЛЕЙ НЕ ИМЕЕТ ИСТИННОГО ЗНАЧЕНИЯ, НО ТОГДА ДОЛЖНО 
ВЫПОЛНЯТЬСЯ УСЛОВИЕ $R$, И ЭТО ПРИВОДИТ НАС КО ВТОРОМУ 
ЛОГИЧЕСКОМУ СЛАГАЕМОМУ; ЛИБО ХОТЯ БЫ ОДИН ПРЕДОХРАНИТЕЛЬ 
ЯВЛЯЕТСЯ ИСТИНОЙ, И ТОГДА СОБЫТИЯ РАЗВИВАЮТСЯ ТАК, КАК ПРИ 
ОДНОКРАТНОМ ЗАПУСКЕ ОПЕРАТОРА "IF" (В НАЧАЛЬНОМ СОСТОЯНИИ, 
КОТОРОЕ НЕ МОЖЕТ ПРИВЕСТИ К НЕМЕДЛЕННОМУ ОТКАЗУ ВСЛЕДСТВИЕ 
ОТСУТСТВИЯ ИСТИННЫХ ЗНАЧЕНИЙ ПРЕДОХРАНИТЕЛЕЙ). оДНАКО ПОСЛЕ 
ТАКОГО ВЫПОЛНЕНИЯ, ПРИ КОТОРОМ ПРОИЗВОДИЛАСЬ ВЫБОРКА ОДНОЙ 
ОХРАНЯЕМОЙ КОМАНДЫ, НАМ НЕОБХОДИМА ГАРАНТИЯ ПОПАДАНИЯ В 
СОСТОЯНИЕ, ТАКОЕ, ЧТОБЫ ПОТРЕБОВАЛОСЬ НЕ БОЛЕЕ
ЧЕМ $(k-1)$ ДАЛЬНЕЙШИХ ВЫБОРОК ДЛЯ ДОСТИЖЕНИЯ ЗАВЕРШЕНИЯ 
РАБОТЫ В КОНЕЧНОМ СОСТОЯНИИ, УДОВЛЕТВОРЯЮЩЕМ $R$. в 
СООТВЕТСТВИИ С НАШИМ ОПРЕДЕЛЕНИЕМ, ТАКИМ ПОСТУСЛОВИЕМ ДЛЯ 
ЭТОГО ОПЕРАТОРА "IF" СЛУЖИТ $H_{k-1}(R)$.

сОГЛАСНО ПОСЛЕДНЕЙ СТРОКЕ, ОПРЕДЕЛЯЮЩЕЙ $\wp(DO, R)$, ДОЛЖНО 
СУЩЕСТВОВАТЬ ТАКОЕ ЗНАЧЕНИЕ $k$, ЧТОБЫ ПОТРЕБОВАЛОСЬ НЕ БОЛЕЕ 
ЧЕМ $k$ ВЫБОРОК ДЛЯ ОБЕСПЕЧЕНИЯ ЗАВЕРШЕНИЯ РАБОТЫ В КОНЕЧНОМ 
СОСТОЯНИИ, УДОВЛЕТВОРЯЮЩЕМ ПОСТУСЛОВИЮ $R$.

{\sl зАМЕЧАНИЕ.} еСЛИ МЫ ДОПУСКАЕМ ТАКЖЕ И ПУСТОЙ НАБОР 
oxРАНЯЕМЫХ КОМАНД, ТО В СИЛУ СКАЗАННОГО ОПЕРАТОР "\kwd{do} \kwd{od}" 
СЕМАНТИЧЕСКИ ЭКВИВАЛЕНТЕН НАШЕМУ ПРЕЖНЕМУ ОПЕРАТОРУ 
"ПРОПУСТИТЬ". {\sl (кОНЕЦ ЗАМЕЧАНИЯ.)}

\bye
